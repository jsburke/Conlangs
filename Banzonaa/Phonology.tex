\documentclass[Banzonaa.tex]{subfiles}

\begin{document}

\section{Phonology}

\subsection{Overview}
Banzonaa's phonology features a vast number of consonants and a small inventory of vowels.  The consonants roughly break into a tenuis, voiced, and ejective series, however the voiced series is fairly scant compared to the other two.  Of further note is the labialization of velar consonants.  The language hoists this onto a fairly simple syllable system.

\subsection{Consonants}
\begin{wrapfigure}{l}{0.65\textwidth}
\begin{tabular}{|l||c|c|c|c|c|c|}
\hline
            & Labial & Alveolar & Post-Alveolar & Lateral & Velar & Labiovelar \\ \hline
Tenuis      & p      & t \za    & \ca           & \tla    & k     & k^w        \\ \hline 
Voiced      & Labial & Alveolar & Post-Alveolar & Lateral & Velar & Labiovelar \\ \hline  
Ejective    & Labial & Alveolar & Post-Alveolar & Lateral & Velar & Labiovelar \\ \hline   
Nasal       & Labial & Alveolar & Post-Alveolar & Lateral & Velar & Labiovelar \\ \hline   
Fricative   & Labial & Alveolar & Post-Alveolar & Lateral & Velar & Labiovelar \\ \hline   
Approximant & Labial & Alveolar & Post-Alveolar & Lateral & Velar & Labiovelar \\ 
\hline
\end{tabular}
\caption{Consonant Inventory}
\end{wrapfigure}

\subsection{Vowels}

\subsection{Syllable Patterns}

\subsection{Environmental Restrictions}

\subsection{Allophones}

\end{document}
