\documentclass[Banzonaa.tex]{subfiles}

\begin{document}

\section{Phonology}

\subsection{Overview}
Nunni features a very small phonemic inventory having only ten sounds.  Most of them surface with multiple realizations.

\subsection{Basic Phonemes}
\begin{wrapfigure}{l}{0.65\textwidth}
\begin{tabular}{|l|l|}
\hline
Phoneme    & Notes                                                              \\ \hline \hline
a - \epsi  & /a/ except for Consonant Breaking \ref{breaking}                   \\
i - j      & \textbf{[j]} for Semivowel Realization \ref{semivowel}             \\
u - w      & \textbf{[w]} for Semivowel Realization \ref{semivowel}             \\
l          & See Sonority \ref{sonority}                                        \\
m          &                                                                    \\
n - \engma & \textbf{[\engma]} before some sounds and word final \ref{nchanges} \\
s - h      & \textbf{[h]} word initially                                        \\
p          &                                                                    \\ 
t          &                                                                    \\ 
k          &                                                                    \\ \hline 
\end{tabular}
\caption{The Ten Phonemes}
\end{wrapfigure}

\subsection{Allophones}

\end{document}
