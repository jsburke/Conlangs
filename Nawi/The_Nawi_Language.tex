\documentclass[10pt]{article}

\usepackage{amsmath}
\usepackage{nth}
\usepackage{graphicx}
\usepackage{wrapfig}
\usepackage{geometry}
	\geometry{letterpaper, left=1.0in, right=1.0in, top=0.9in, bottom=1.0in}

\usepackage[colorlinks=true, allcolors=blue]{hyperref}

\title{The Nawi Language}
\date{\today}
\author{John Sarris Burke}

\graphicspath{{./glyphs/}{./images/}}


\begin{document}
\maketitle

\tableofcontents

\pagebreak

\pagebreak

\section{Introduction}

\paragraph{Inspiration}
Nawi is an {\em a priori} constructed language by John Burke to explore topics from natural languages.  The phonology draws from Japanese, Burmese, and Cree.  The grammar is highly analytic with a handful of synthetic elements.  It has a strong basis in Chinese and English with influences from Sumerian, Nahuatl, and several other languages.

\section{Phonology}
	\subsection{Vowels}
	\subsection{Consonants}
	\subsection{Allophones}
	\subsection{Ellision}
\section{The Nawi Script}
\section{Grammar}
\section{Pronouns}
	\subsection{Personal Pronouns}
	\subsection{Demonstrative Pronouns}
	\subsection{Reflexive Pronouns}
	\subsection{Interrogative Pronouns}
	\subsection{Indefinite Pronouns}
\section{Nouns}
	\subsection{Possession}
		\subsubsection{Alienable}
		\subsubsection{Inalienable}
		\subsubsection{Personal Pronouns}
	\subsection{Noun Phrases}
	\subsection{Adjectival Phrases}
\section{Verbs}
	\subsection{Tense}
	\subsection{Aspect}
	\subsection{Modalities}
\section{Syntax and Sentence Structure}
		


\pagebreak

\begin{thebibliography}{99}
	
    
\end{thebibliography}

\end{document}