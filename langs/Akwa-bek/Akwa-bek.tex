\documentclass[11pt,letterpaper]{article}

\title{The Akwa Bek Language}
\date{\today}
\author{John Sarris Burke}

%commands for spacing and polish
\newcommand{\setmargin}{0.8in}
\newcommand{\pseudoitem}{\diamond}
\newcommand{\vertspace}{\vspace{1.2mm}}
\newcommand{\horzindent}{\hspace{3mm}}
\newcommand{\minorindent}{\hspace{4.5mm}}
\newcommand{\minoritem}{\minorindent $\pseudoitem$ }
\newcommand{\latex}{\LaTeX\space}

\usepackage{geometry}
	\geometry{left=\setmargin, right=\setmargin, top=\setmargin, bottom=\setmargin}

\usepackage[colorlinks=true, allcolors=blue]{hyperref}
\usepackage{subfigure}
\usepackage{nth}
\usepackage{graphicx}
\usepackage{wrapfig}
\usepackage{amsmath}
\usepackage{textcomp}

%IPA I will need
\usepackage{tipa}

% vowel sounds
\newcommand{\tschwa}{\textipa{@}}
\newcommand{\tlonga}{\textipa{a:}}
\newcommand{\tlongi}{\textipa{i:}}
\newcommand{\tlongo}{\textipa{o:}}
\newcommand{\tlonge}{\textipa{e:}}
\newcommand{\tlongu}{\textipa{u:}}

% consonants
\newcommand{\tx}{\textipa{S}}
\newcommand{\tj}{\textipa{\textbeltl}}

\newcommand{\tz}{\textipa{\t{ts}}}
\newcommand{\tc}{\textipa{\t{t\tx}}}
\newcommand{\tq}{\textipa{\t{t\tj}}}

\newcommand{\tkw}{\textipa{k\super w}}
\newcommand{\txw}{\textipa{x\super w}}

\newcommand{\jj}{\textipa{Z}}
\newcommand{\yoghlig}{\textipa{\textlyoghlig}}
\newcommand{\tgg}{\textipa{G}}
\newcommand{\tggw}{\textipa{\tgg\super w}}
\newcommand{\cs}{\textipa{\c{c}}}


\begin{document}
\maketitle
\tableofcontents
\pagebreak
\pagebreak

\section{Preface}
  Akwa Bek, literally "God's Language" is a constructed language that is the synthesis of years of casual study, exploration of interesting linguistic features, a pursuit of realism, and an attempt to express Christian thinking and expression. Constructed languages, also commonly called conlangs by those who study and produce them, are an interesting topic with a longer history than many expect. To those unaccustomed to them, they seem odd, and the hobby as it is seen in the modern day can be confusing. However, they have gained some mainstream clout from popular culture; The Lord of the Rings and Game of Thrones will be quick references for many. Esperanto may also be recognized by many as one of the attempts at making and International Auxiliary Language (and \textit{ial} in the conlang community). My motivation for producing Akwa Bek is partially as a hobby and partially to explore a strong Christian influence in an \textit{a priori} conlang, one that is not directly derived from existing languages.
  \par
  I had started conlanging at a relatively young age after being introduced to Latin and later Japanese. For some reason, like many other conlangers, I was drawn to start drafting something of my own without knowing a community was already present and starting to congregate in online communities. While many of the world's languages are indeed an inspiration for Akwa Bek, it is much moreso the case that my peers are my motivation. I have no expectation that any of them, let alone myself, will ever gain fluency in this language; however, it is my sincere hope that this book will provide inspiration and a worthwhile diversion for others in the conlang community.
  \par
  Akwa Bek will primarily be an experiment therefore to pull together linguistic elements that I like or find interesting in addition to looking into how a language driven by Christian thought and theology may resolve. That said, a large motivation behind this will be charity, and I hope to capture even a fragment of the Charity of Christ in this.
  \par
  All this said, I don't want to muddy the language with my mental meandering all that much, so enjoy your reading!
\pagebreak

%%%%%%%%%%%%%%%%%%%%%%%%%%%%%
%%                         %%
%% Phonology               %%
%%                         %%
%%%%%%%%%%%%%%%%%%%%%%%%%%%%%

\section{Phonology}
\label{phonology}
  \subsection{Overview}
  Akwa Bek has a relatively simple sound system. There are four vowel qualities, each in two lengths, and nineteen consonant sounds. The syllable structure is moderately complex, but the majority of that complexity arises at the start of words. Like any language, there is some variation in how these sounds are realized depending on its environment.	

  \subsection{Vowels}
  \label{vowels}
     \begin{wrapfigure}{l}{0.5\textwidth}
       \begin{tabular}{|l|l|l|l|}
         \hline
                 & Front      & Central    & Back         \\ \hline \hline
         High    & i, \tlongi &            & (u, \tlongu) \\
         Mid     & e, \tlonge & (\tschwa)  & o, \tlongo   \\
         Low     &            & a, \tlonga &              \\ \hline
       \end{tabular}
     \end{wrapfigure}
     \par
     Akwa Bek has four cardinal vowels that arise in long and short varieties; they are nonparenthetic in the table to the left. The [u] sounds and [\tschwa] arise from environmental factors discussed in \hyperref[sec:soundalternations]{sound changes section}. There are no diphthongs in the language. Instead, simple sequences of vowels are present.

  \subsection{Consonants}
  \label{consonants}
     \begin{wrapfigure}{r}{0.65\textwidth}
       \begin{tabular}{|l|l|l|l|l|l|}
         \hline
                     & Bilabial   & Alveolar & Palatal & Lateral & Velar     \\ \hline \hline
         Voiceless   & p          & t        &         &         & k, \tkw   \\
         Voiceed     & b          & d        &         &         &           \\
         Nasal       & m          & n        &         &         &           \\
         Fricative   &            & s        & \tx     &  \tj    & x, \txw   \\
         Affricate   &            & \tz      & \tc     &  \tq    &           \\
         Approximant &            &          & j       &  l      & w         \\ \hline
       \end{tabular}
     \end{wrapfigure}
     \par
     Akwa Bek has nineteen distinct consonant sounds; however the vast majority can only occur at the beginning of a syllable. Furthermore, all consonants may be geminated with the exception of the approximants and voiced stops.
     \par
     Consonants that can appear at the end of a syllable are much more restricted. Only /l/, /k/, /t/, a homorganic nasal /N/, and a "chroneme" /Q/ may appear in coda position. Of these, /k/ and /t/ only occur at the end of a word, and the "chroneme" never is at the end of a word. /Q/ essentially means that the consonant is geminate, such that /ita/, \textit{field mouse}, is distinct from /i\textipa{t:}a/, \textit{bone}.

  \subsection{Phonotactics}
  \label{phonotactics}

    \subsubsection{Syllable Structure}
    \label{syllables}
    The syllable structure is broadly (s/\tx)(C)V(k/t/l/Q/N). The initial /s/ or /\tx/ are only present at the start of words, and they never form clusters with other fricatives in that position. In the coda position /k/ and /t/ are only ever present at the end of a word, and /Q/ is only ever present word medially. Both /l/ and /N/ may appear freely at the end of any syllable, but /l/ is uncommon in the word final position.

    \subsubsection{Timing}
    \label{timing}
    \begin{wrapfigure}{r}{0.45\textwidth}
      \begin{tabular}{|c|c|}
      \hline
      Syllable          & Morae Count \\ \hline \hline
      (s/\tx)(C)V       & 1           \\
      (s/\tx)(C)V:      & 2           \\
      (s/\tx)(C)V(l/N)  & 2           \\
      (s/\tx)(C)V:(l/N) & 3           \\ \hline
      \end{tabular}
    \end{wrapfigure}
    Akwa Bek is a mora timed language. The table here provides examples of how various syllables map to morae counts. Generally, syllables with a short vowel are one mora. Those with a long vowel are two morae. Syllable final /l/, /N/, and /Q/ constitute a mora as well, however word final /k/ and /t/ do not. 

  \subsection{Sound Alternations}
  \label{soundalternations}
  Like any other language, the sounds in Akwa Bek are not invariable. Depending on their environment, they will sound different than the raw phonemes enumerated in the tables above.

    \subsubsection{Intervocalic Lenition}
    \label{lenition}
    Between vowels, /b/ and /d/ will weaken to [v] and [r], respectively. This applies across word boundaries, not only within words. A simple example of this is in the name of the language itself, which is realized as [a\tkw a vek].

    \subsubsection{Palatalization}
    \label{palatalization}
    When /k/ or /x/ appear before /i/ or /e/, they will palatalize to [c] and [\cs] respectively. For word final /k/, this process does not occur. This pattern is common to long and short /i/ and /e/.

    \subsubsection{Assimilation}
    \label{assimilation}
    A decent number of sounds are subject to assimilation. Firstly, the fricatives will voice to [z, \jj, \yoghlig, \tgg, \tggw] when not geminated in word medial positions. For /x/ in particular, this can compound with palatalizing which will result in [\textipa{J}]. This voicing never occurs across word boundaries.
    \par
    Another process of assimilation happens when /w/, /\tkw/, and /\txw/ are before /o/. When this happens, the resulting syllables are [u], [ku], and [xu], respectively. This same pattern applies to long and short /o/.

    \subsubsection{Epenthesis}
    \label{epenthesis}
    While relatively rare, across word boundaries, clusters of three consonants can occur. However, Akwa Bek does not tolerate clusters of more than two consonants anywhere. When this occurs, and epenthetic vowel, [\tschwa], will be placed in between the two words. This epenthetic vowel adds a single mora to the utterance.

\pagebreak

%%%%%%%%%%%%%%%%%%%%%%%%%%%%%
%%                         %%
%% Latinization            %%
%%                         %%
%%%%%%%%%%%%%%%%%%%%%%%%%%%%%

\section{Latinization}
\label{latinization}
Akwa Bek has its own orthography that will be discussed later, but for ease of description, a Latinization scheme is provided here. It is broadly phonemic, but loosens itself a little bit for simplicity. In general, there are no rules for capitalization, so capitals may be used in an \textit{ad hoc} manner.

  \subsection{Vowels}
  The vowels map directly to their IPA glyphs, with the exception on the epenthetic schwa. Short vowels are a single letter: <a, e, o, i>. Long vowels are graphically doubled: <aa, ee, oo, ii>. When [u] arises from /o/ and /\tlongo/ following /w/ like sounds as noted above, the latinization relaxes away from a phonemic style and takes a phonetic approach such that /\tkw o/ will be Latinized as <ku>. The use of <u> doubles in the same way as other vowel graphs. The epenthetic schwa, when it arises, is represented with <a> attaching to the second word.

  \subsection{Consonants}
  \begin{wrapfigure}{r}{0.65\textwidth}
    \begin{tabular}{|l|l|l|l|l|l|}
      \hline
                  & Bilabial   & Alveolar & Palatal & Lateral & Velar     \\ \hline \hline
      Voiceless   & p          & t        &         &         & k, kw     \\
      Voiceed     & b          & d        &         &         &           \\
      Nasal       & m          & n        &         &         &           \\
      Fricative   &            & s        & x       &  j      & h, hw     \\
      Affricate   &            & z        & c       &  q      &           \\
      Approximant &            &          & j       &  l      & w         \\ \hline
    \end{tabular}
  \end{wrapfigure}
  The table here presents the latinization of the consonants. The homorganic nasal /N/ is simply written as <n>, regardless of realization, and the chroneme /Q/ is represented by doubling the consonant. As noted in the vowel section for latinizing, the <w> will be dropped before any <o> and both represented by a <u>. Writers may elect to note /b/ and /d/ as <v> and <r> respectively when they are lenited, though this is not strict.

\pagebreak
  
%%%%%%%%%%%%%%%%%%%%%%%%%%%%%
%%                         %%
%% Grammar                 %%
%%                         %%
%%%%%%%%%%%%%%%%%%%%%%%%%%%%%

\section{Grammar}
\label{grammar}
Akwa Bek has a relatively analytic grammar relying on word and phrase order to express many relationships. That said, there are several fused elements that play within that larger context, and there are a few synthetic processes as well. The language veers towards the noun heavy side in general. Most cases where English would employ an adjective, Akwa Bek uses a noun, possibly with some phrasal garnish. Auxiliary verbs get fairly heavy use as well.

  \subsection{Lexical Categories}
  \label{categories}
  Akwa Bek has the following lexical classes: Adnominals, Verbs, Classifiers, Conjunctions, and Bound Particles. Adnominals roughly map to nouns, adjectives, and many adverbs in English. Verbs generally map with verbs in English and some adjectives. Classifiers cover determiners and other classes of words. Conjuntions help string utterances together and often cover discourse particles in some languages. Finally Bound Particles are elements that help words relate to each other in specific ways.

  \subsection{Adnominals}
  \label{adnominals}
  Adnominals cover classes from nouns, to adjectives to adverbs. They are loosely broken into two groups, animate and inanimate. Animacy affects how they interact with verbs and how they are refered to by other groups like pronouns and classifiers. In general, adnominals that "feel" more like an adjective in the English sense will be inanimate. Among things that appear more traditionally as nouns, animals, some plants, things seen as representing the holy, and so on are regularly animate, while things like minerals, tools, cloth, and that related to the unholy, such as sin, are inanimate.

    \subsubsection{Pronouns}
    \label{pronouns}
    \begin{wrapfigure}{l}{0.45\textwidth}
      \begin{tabular}{|l|l|l|}
        \hline
                            & Singular  & Plural \\ \hline \hline
        First               & i         & ei     \\
        Second              & ne        & ini    \\
        Animate             & ka        & ki     \\
        Animate Obviative   & kee       & koi    \\
        Inanimate           & awaa      & awaa   \\ \hline
      \end{tabular}
    \end{wrapfigure}
    The third person pronouns only distinguish on the basis of animacy, though animate nouns are marked for obviation. Obviation will be discussed in more detail later, but it is used to distinguish third person animate antecedants in the case of ambiguity. The most common cases are a third person subject and objects, where in following statements the object antecedants, either direct or indirect, would be referred to with the obviative pronoun. Another common case is a prior statement having a noun phrase that showed possession or attribution. The possessor or attributor would be referred to via the obviative.
    \par
    A further thing to note is that the inanimate has no distinct plural pronoun. This is the general case in Akwa Bek. Inanimate adnominals do not pluralize. For countable things, numbers can be specified. For uncountable things, especially the unholy, no need to make them more numerous.

\end{document}
