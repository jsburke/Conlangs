\documentclass[11pt,letterpaper]{article}

\title{The Akwa Bek Language}
\date{\today}
\author{John Sarris Burke}

%commands for spacing and polish
\newcommand{\setmargin}{0.8in}
\newcommand{\pseudoitem}{\diamond}
\newcommand{\vertspace}{\vspace{1.2mm}}
\newcommand{\horzindent}{\hspace{3mm}}
\newcommand{\minorindent}{\hspace{4.5mm}}
\newcommand{\minoritem}{\minorindent $\pseudoitem$ }
\newcommand{\latex}{\LaTeX\space}

\usepackage{geometry}
	\geometry{left=\setmargin, right=\setmargin, top=\setmargin, bottom=\setmargin}

\usepackage[colorlinks=true, allcolors=blue]{hyperref}
\usepackage{subfigure}
\usepackage{nth}
\usepackage{graphicx}
\usepackage{wrapfig}
\usepackage{amsmath}
\usepackage{textcomp}

%IPA I will need
\usepackage{tipa}

% vowel sounds
\newcommand{\tschwa}{\textipa{@}}
\newcommand{\tlonga}{\textipa{a:}}
\newcommand{\tlongi}{\textipa{i:}}
\newcommand{\tlongo}{\textipa{o:}}
\newcommand{\tlonge}{\textipa{e:}}
\newcommand{\tlongu}{\textipa{u:}}

% consonants
\newcommand{\tx}{\textipa{S}}
\newcommand{\tj}{\textipa{\textbeltl}}

\newcommand{\tz}{\textipa{\t{ts}}}
\newcommand{\tc}{\textipa{\t{t\tx}}}
\newcommand{\tq}{\textipa{\t{t\tj}}}

\newcommand{\tkw}{\textipa{k\super w}}
\newcommand{\txw}{\textipa{x\super w}}

\newcommand{\jj}{\textipa{Z}}
\newcommand{\yoghlig}{\textipa{\textlyoghlig}}
\newcommand{\tgg}{\textipa{G}}
\newcommand{\tggw}{\textipa{\tgg\super w}}
\newcommand{\cs}{\textipa{\c{c}}}


\begin{document}
\maketitle
\tableofcontents
\pagebreak
\pagebreak

%%%%%%%%%%%%%%%%%%%%%%%%%%%%%
%%                         %%
%% Preface and Meta        %%
%%                         %%
%%%%%%%%%%%%%%%%%%%%%%%%%%%%%

\section{Preface}
  Akwa Bek, literally "God's Language", is a constructed language meant to explore a handful of linguistic topics while creating something that is elegant and potentially useful. The topics of investigated cover as many features in a language as possible, ranging from the sounds, to grammar, and the lexicon. I plan on looking into zero-derivation, over zealous applications of possessives, and a rather unnatural division in lexical categories that will place verbs in one context and nouns, adjectives, and adverbs in another. I also hope to leverage contractions and phrase level clitics, both topics I see seldom addressed in conlangs, \textit{a priori} conlangs of any stripe in particular. When it comes to those points where I encounter an odd or unforseen decision for Akwa Bek, there is no intent for any methodology aside from what feels cohesive for the rest of the language.
\par
  Unlike many constructed languages, Akwa Bek is not starting with any fantasy world or culture in mind, nor is it explicitly intended for auxiliary use like Esperanto or Lojban. As the author, I understand fully that in a way this can present a hurdle to developing a sense of naturalness or realism, and I accept that this fundamental choice presents this challenge. I am simply not interested in those topics for Akwa Bek. I do hope to weave some semblance of a Christian cosmovision through the lexicon, collocations, and idioms, but otherwise it can sit in a modern setting. That said, I only want that mindset to go so far; I would far prefer that it acts more like the spices in a meal than the meat or wine. Enough to be a pleasant diversion for those who care to look in the lexicon or eventual corpus. I am making no attempt in Akwa Bek to extend such a cosmovision to the grammar; I don't doubt that it is possible, but such heavy handedness is not desired here.
\par
  Beyond these things, I have a few other simple goals for Akwa Bek. I would like it to be as complete as a conlang can truly be without having a wide community of speakers. Namely, I want a strong enough grammar and wide enough lexicon that the only future actions needed would be extrapolating or adding to the lexicon. It would be pleasant to try to cultivate a community around it, but this truly is a stretch goal for any conlang.
\par
  Finally, I would simply like that you, the reader, enjoy Akwa Bek as fully as possible, regardless of how you have come to find this description. With that said, thank you for taking the time to read this.
\pagebreak

\section{Change Log}
  Here will be the record of major changes to the Akwa Bek language description over time. I do not plan on enumerating anything here until I have at least a comprehensive rough first draft.
\pagebreak

%%%%%%%%%%%%%%%%%%%%%%%%%%%%%
%%                         %%
%% Phonology               %%
%%                         %%
%%%%%%%%%%%%%%%%%%%%%%%%%%%%%

\section{Phonology}
\label{phonology}
  \subsection{Overview}
  The Akwa Bek Phonological system is fairly modest featuring a few minorly interesting gaps, and a couple inclusions that while not very odd but might feel exotic to many people, especially more accustomed to languages from Europe. The phonotactics are similarly uncomplicated, somewhat reminiscent of Japanese. The patterns of synchronic sound alterations is also fairly simple since phonetics is not a high priority interest, but it may develop with time for the sake of naturalness or ease of speaking.
\vspace{5mm}

  \subsection{Vowels}
  \label{vowels}
     \begin{wrapfigure}{l}{0.5\textwidth}
       \begin{tabular}{|l|l|l|l|}
         \hline
                 & Front      & Central    & Back         \\ \hline \hline
         High    & i          &            & (u)          \\
         Mid     & e          &            & o            \\
         Low     &            & a          &              \\ \hline
       \end{tabular}
     \end{wrapfigure}
     Akwa Bek has a four vowel inventory with some flexibility because of the syllable structure. What superficially look like long vowels and diphthongs are truly vowel clusters. Finally, [u] arises when /o/ follows a labialized consonant. When [u] does appear in this way, subsequent /o/ vowels will also raise to [u], but this does not extend across consonants or word boundaries.
  \par
\vspace{5mm}

  \subsection{Consonants}
  \label{consonants}
       \begin{tabular}{|l|l|l|l|l|l|}
         \hline
                     & Bilabial   & Alveolar & Palatal & Lateral & Velar     \\ \hline \hline
         Voiceless   & p          & t        &         &         & k, \tkw   \\
         Voiceed     & b$^1$      & d$^2$    &         &         &           \\
         Nasal       & m          & n        &         &         &           \\
         Fricative   & v$^1$      & s        & \tx     &  \tj    & x, \txw   \\
         Affricate   &            & \tz      & \tc     &  \tq    &           \\
         Approximant &            &          & j       &  l      & w         \\ 
         Flap        &            & r$^2$    &         &         &           \\ \hline
       \end{tabular}
     \vspace{5mm}
     \par
     Akwa Bek features nineteen distinct phonemes. It notably lacks /g/ and /f/, and less notably lacks velar nasals which may be expected. The velars come with a labialization distinction, and the voiced obstruents are rapidly weakened. Most of the other consonants are relatively stable, but some alterations and shifts occur in fast speech.

  \subsection{Phonotactics}
  \label{phonotactics}

    \subsubsection{Syllable Structure}
    \label{syllables}
    The syllable structure is broadly (s/\tx)(C)V(k/t/l/Q/N). The initial /s/ or /\tx/ are only present at the start of words, and they never form clusters with other fricatives in that position. In the coda position /k/ and /t/ are only ever present at the end of a word, and /Q/ is only ever present word medially. Both /l/ and /N/ may appear freely at the end of any syllable, but /l/ is uncommon in the word final position.

    \subsubsection{Timing}
    \label{timing}
    \begin{wrapfigure}{r}{0.45\textwidth}
      \begin{tabular}{|c|c|}
      \hline
      Syllable          & Morae Count \\ \hline \hline
      (s/\tx)(C)V       & 1           \\
      (s/\tx)(C)V:      & 2           \\
      (s/\tx)(C)V(l/N)  & 2           \\
      (s/\tx)(C)V:(l/N) & 3           \\ \hline
      \end{tabular}
    \end{wrapfigure}
    Akwa Bek is a mora timed language. The table here provides examples of how various syllables map to morae counts. Generally, syllables with a short vowel are one mora. Those with a long vowel are two morae. Syllable final /l/, /N/, and /Q/ constitute a mora as well, however word final /k/ and /t/ do not. 

  \subsection{Sound Alternations}
  \label{soundalternations}
  Like any other language, the sounds in Akwa Bek are not invariable. Depending on their environment, they will sound different than the raw phonemes enumerated in the tables above.

    \subsubsection{Intervocalic Lenition}
    \label{lenition}
    Between vowels, /b/ and /d/ will weaken to [v] and [r], respectively. This applies across word boundaries, not only within words. A simple example of this is in the name of the language itself, which is realized as [a\tkw a vek].

    \subsubsection{Palatalization}
    \label{palatalization}
    When /k/ or /x/ appear before /i/ or /e/, they will palatalize to [c] and [\cs] respectively. For word final /k/, this process does not occur. This pattern is common to long and short /i/ and /e/.

    \subsubsection{Assimilation}
    \label{assimilation}
    A decent number of sounds are subject to assimilation. Firstly, the fricatives will voice to [z, \jj, \yoghlig, \tgg, \tggw] when not geminated in word medial positions. For /x/ in particular, this can compound with palatalizing which will result in [\textipa{J}]. This voicing never occurs across word boundaries.
    \par
    Another process of assimilation happens when /w/, /\tkw/, and /\txw/ are before /o/. When this happens, the resulting syllables are [u], [ku], and [xu], respectively. This same pattern applies to long and short /o/.

    \subsubsection{Epenthesis}
    \label{epenthesis}
    While relatively rare, across word boundaries, clusters of three consonants can occur. However, Akwa Bek does not tolerate clusters of more than two consonants anywhere. When this occurs, and epenthetic vowel, [\tschwa], will be placed in between the two words. This epenthetic vowel adds a single mora to the utterance.

\pagebreak

%%%%%%%%%%%%%%%%%%%%%%%%%%%%%
%%                         %%
%% Latinization            %%
%%                         %%
%%%%%%%%%%%%%%%%%%%%%%%%%%%%%

\section{Latinization}
\label{latinization}
Akwa Bek has its own orthography that will be discussed later, but for ease of description, a Latinization scheme is provided here. It is broadly phonemic, but loosens itself a little bit for simplicity. In general, there are no rules for capitalization, so capitals may be used in an \textit{ad hoc} manner.

  \subsection{Vowels}
  The vowels map directly to their IPA glyphs, with the exception on the epenthetic schwa. Short vowels are a single letter: <a, e, o, i>. Long vowels are graphically doubled: <aa, ee, oo, ii>. When [u] arises from /o/ and /\tlongo/ following /w/ like sounds as noted above, the latinization relaxes away from a phonemic style and takes a phonetic approach such that /\tkw o/ will be Latinized as <ku>. The use of <u> doubles in the same way as other vowel graphs. The epenthetic schwa, when it arises, is represented with <a> attaching to the second word.

  \subsection{Consonants}
  \begin{wrapfigure}{r}{0.65\textwidth}
    \begin{tabular}{|l|l|l|l|l|l|}
      \hline
                  & Bilabial   & Alveolar & Palatal & Lateral & Velar     \\ \hline \hline
      Voiceless   & p          & t        &         &         & k, kw     \\
      Voiceed     & b          & d        &         &         &           \\
      Nasal       & m          & n        &         &         &           \\
      Fricative   &            & s        & x       &  j      & h, hw     \\
      Affricate   &            & z        & c       &  q      &           \\
      Approximant &            &          & j       &  l      & w         \\ \hline
    \end{tabular}
  \end{wrapfigure}
  The table here presents the latinization of the consonants. The homorganic nasal /N/ is simply written as <n>, regardless of realization, and the chroneme /Q/ is represented by doubling the consonant. As noted in the vowel section for latinizing, the <w> will be dropped before any <o> and both represented by a <u>. Writers may elect to note /b/ and /d/ as <v> and <r> respectively when they are lenited, though this is not strict.

\pagebreak
  
%%%%%%%%%%%%%%%%%%%%%%%%%%%%%
%%                         %%
%% Grammar                 %%
%%                         %%
%%%%%%%%%%%%%%%%%%%%%%%%%%%%%

\section{Grammar}
\label{grammar}
Akwa Bek has a relatively analytic grammar relying on word and phrase order to express many relationships. That said, there are several fused elements that play within that larger context, and there are a few synthetic processes as well. The language veers towards the noun heavy side in general. Most cases where English would employ an adjective, Akwa Bek uses a noun, possibly with some phrasal garnish. Auxiliary verbs get fairly heavy use as well.

  \subsection{Lexical Categories}
  \label{categories}
  Akwa Bek has the following lexical classes: Adnominals, Verbs, Classifiers, Conjunctions, and Bound Particles. Adnominals roughly map to nouns, adjectives, and many adverbs in English. Verbs generally map with verbs in English and some adjectives. Classifiers cover determiners and other classes of words. Conjuntions help string utterances together and often cover discourse particles in some languages. Finally Bound Particles are elements that help words relate to each other in specific ways.

  \subsection{Adnominals}
  \label{adnominals}
  Adnominals cover classes from nouns, to adjectives to adverbs. They are loosely broken into two groups, animate and inanimate. Animacy affects how they interact with verbs and how they are refered to by other groups like pronouns and classifiers. In general, adnominals that "feel" more like an adjective in the English sense will be inanimate. Among things that appear more traditionally as nouns, animals, some plants, things seen as representing the holy, and so on are regularly animate, while things like minerals, tools, cloth, and that related to the unholy, such as sin, are inanimate.

    \subsubsection{Pronouns}
    \label{pronouns}
    \begin{wrapfigure}{l}{0.45\textwidth}
      \begin{tabular}{|l|l|l|}
        \hline
                            & Singular  & Plural \\ \hline \hline
        First               & i         & ei     \\
        Second              & ne        & ini    \\
        Animate             & ka        & ki     \\
        Animate Obviative   & kee       & koi    \\
        Inanimate           & awaa      & awaa   \\ \hline
      \end{tabular}
    \end{wrapfigure}
    The third person pronouns only distinguish on the basis of animacy, though animate nouns are marked for obviation. Obviation will be discussed in more detail later, but it is used to distinguish third person animate antecedants in the case of ambiguity. The most common cases are a third person subject and objects, where in following statements the object antecedants, either direct or indirect, would be referred to with the obviative pronoun. Another common case is a prior statement having a noun phrase that showed possession or attribution. The possessor or attributor would be referred to via the obviative.
    \par
    A further thing to note is that the inanimate has no distinct plural pronoun. This is the general case in Akwa Bek. Inanimate adnominals do not pluralize. For countable things, numbers can be specified. For uncountable things, especially the unholy, no need to make them more numerous.

  \subsection{Bound Particles}
  \label{boundparticles}
  Bound particles are a group of morphemes in Akwa Bek that attach to other words, most frequently adnominals and verbs, to indicate changes in meaning, usage, or to indicate how words or phrases interact. There are no hard and fast rules to how they bind to other elements, so much of this section will describe how they behave in addition to what they indicate.

    \subsubsection{Possessive and Attributive Particles}
    \label{attributive}
    Akwa Bek groups possession and attribution under the same roof, and distinguishes them in the same way. There are two particles that serve this purpose: \textbf{s(e)-} and \textbf{-(a)k}. The first is called the extrinsic attributive, and the second is the intrinsic. In the sense of possession, this roughly aligns with alienable and inalienable possession. In the sense of general attribution, it deals with descriptions that are temporary or fleeting versus permanent.
    \par
    \subsubsection{Intrinsic Attribution}
    \label{intrinsic}
    Intrinsic attribution, and inalienable possession, describe relationships that are perceived as permanent. It is marked on the attributed adnominal by suffixing the \textbf{-(a)k} particle. It surfaces as \textbf{ak} if the word ends in any consonant, and \textbf{k} if it ends in a vowel. It does not add an additional mora to the attached noun. A simple example in the adjectival sense:
    \par
    \texttt{coo damak} - a dark night - literally: \textit{darkness night-INTR}
    \par
    In the sense of inalienable possession, it is used to show family relationships, one's origin, and things not easily taken away, like body parts. And example of this usage:
    \par
    \texttt{bawa ittak} - the bone from a dog -- \textit{dog bone-INTR}
    \par
    This differs from the alienable version, \texttt{sebawa itta}, which would be understood to be a bone given to a dog as a snack for example, while the example with \texttt{ittak} would refer to somethning like its thigh or spine.
    \par
    \subsubsection{Extrinsic Attribution}
    \label{extrinsic}

\end{document}
