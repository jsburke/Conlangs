\section{Grammatical Overview}
This section tries to provide a high level view of \langname rather than a more fully fleshed out view that will follow. It covers concepts like alignment, reconstructed verb uses and categories, word order, and so on. Sections following this one will give the morphology in much greater depth on a particle by particle and usage by usage basis. Those sections, despite the translations and glosses provided, may require multiple readings of the text or referencing other parts for better understanding.
\subsection{Nominative Alignment}
\langname is a completely nominative language, and frequently the nominative will go unmarked while the accusative is marked. However, this can flip when subjects that normally cannot serve as a subject do so, especially in figurative uses or phrases. There is an implied animacy or volitionality that drives this it seems; this fluidity in marking is lost via case calcification or fusion in daughter languages.
\subsubsection{Case Suffixes}
There are some 9 case suffixes for indicating different roles in an utterance. It seems that for roles not covered by these cases, the genetive and accusative were often employed in more round about ways to extend meaning where the cases could not supply.
\subsection{Word Order}
\langname seems to have strongly preferred \texttt{SOV} word order; however, free word order seems possible so long as verbs remained thoroughly final. This even seems to be the point of syntactic extension of case where a verb of some kind was used to show some role via the accusative but the verb would still follow the accusativized noun.
\subsection{Postpositional Creep}
While postpositions were truly not known at the period of this reconstruction it seems, the usage of genetive and accusative work arounds for what would be prepositions in English seems to have been the source of postpositions in daughter languages which show a great variety of them.
