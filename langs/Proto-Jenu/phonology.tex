\section{Phonology}
The current reconstruction of \langname presents both a small vowel and consonant inventory, with a moderately complex syllable structure. Here we present the phonemes that seem to be present. Within the sections enumerating the phonemes, the Romanization will be in angle brackets to the right after the phoneme.
  \subsection{Vowels}
     \begin{wrapfigure}{r}{0.55\textwidth}
       \begin{tabular}{|l|l|l|l|}
         \hline
                 & Front & Central & Back  \\ \hline \hline
         High    & i <i> &         &       \\ \hline
         Mid     & e <e> &         & o <o> \\ \hline
         Low     &       & a <a>   &       \\ \hline
       \end{tabular}
     \end{wrapfigure}
     There are four vowel qualities that seem to show no difference in length, tone, or any other feature. This forms a rough square in the space. There may have been conditions, such as if a syllable was open or closed, surrounding voicing, and so on, that caused the qualities of the vowels to shift in certain ways.
  \subsection{Consonants}
    \begin{center}
    \begin{tabular}{|l|l|l|l|l|}
      \hline
                    & Labial      & Alveolar &  Palatal            & Velar \\ \hline \hline
        Plain       & p <p>       & t <t>    &  \ipach <c>         & k <k> \\ \hline 
        Voiced      & b <b>       & d <d>    &                     &       \\ \hline 
        Nasal       & m <m>       & n <n>    &                     &       \\ \hline 
        Fricative   & f <f>       & s <s>    &                     &       \\ \hline 
        Approximant &             & l~r <l>  &  j <j>              & w <w> \\ \hline
    \end{tabular}
    \end{center}
  \par\par
  \langname is reconstructred with 13 consonant sounds, most at the labial or alveolar locations. There is a voicing distinction that is not fuly mirrored in the series, and the nasals likewise prefer more common sounds, avoiding anything further back in the mouth and throut. The alveolar approximant cannot be convincingly reconstructed as either a rhotic or a lateral, so it is presented here as varying between the two though this seems not to have been the case as attested in its unusual patterning in daughter languages. 
  \subsection{Syllable Structre}
  The canonical syllable structure is \texttt{(C)V(K)(s)} where \texttt{K} is any consonant except \phonemic{\ipach, b, d, j, w}. There is also an implicit rule that surfaces with affixes that the maximum consonant cluster is 2 consonants long. Vowel - Vowel sequences can occur with a \texttt{CV.V} syllable sequence, and in this case it will sound twice as long as a \texttt{CV} syllable.
  \subsection{Allophony}
  It is suspected that some synchronic sound changes surfaced because of environment in \langname; however, we cannot know for sure. The following sections enumerate what is expected given the relatively small inventory of phonemes overall.
    \subsubsection{Palatalizing of \phonemic{n}}
    /n/ probably surfaced as [\textipa{\textltailn}] before \phonemic{i, e}.
    \subsubsection{Velarization of \phonemic{n}}
    /n/ probably surfaced as [\textipa{N}] before \phonemic{k}. When this was followed by yet another vowel, it is likely that the \phonemic{k} became \phonetic{g}. 
    \subsubsection{Interruptive Glottal Stop}
    While not present as a distinct phoneme, it is expected that the glottal stop, [\textipa{P}], was inserted to break vowel-vowel sequences across word boundaries.
  \subsection{Stress}
  Stress was predictable but not phonemic in \langname. When a word ends with a vowel, the stress falls on the penultimate syllable, the second to last syllable. When the word ends with a consonant then the stress is on the ultimate, final, syllable. Morphological suffixes, such as case markers or plural marking, and derivational affixes can cause this stress to shift away from the syllable that would be assumed by a dictionary entry. It is clear that some daughter languages had some vowel loss because of this stress shifting.
