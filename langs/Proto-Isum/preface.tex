\section{Preface}
\subsection{Motivations and Context}
  \langname is the progenitor of several languages in the \worldname\space universe. It serves as a mean for deriving further languages that will be more fully fleshed out and utilized, and it will feed other conlangs in universe via loanwords, sprachbunds, and possibly creoles, mixed languages, or pidgins. The speakers of the langage, who call themselves \langsample{sulaa} which simply means \transsample{person}. Details on their lives and cultures are elaborated in the section that follows this one.\par
  The sound system of \langname isn't derivative of any particular natural language, but was meant more as an experience in forming a phonology in a strongly a priori fashion. To this end, the vowels were envisioned as two overlying triangles: a stronger one composed of \phonemic{a, i, u} with both long and short forms with a weaker \phonemic{\tshorty, \tshorte, \tshorto} with only one length. It is abundantly clear that this idea of two overlayed triangles in the vowel bears no rigors of linguistics or scientific understanding of language; it is simply something that felt reasonably natural and was appealing enough to commit to. The large consonant inventory was very loosly based on older Indo-European languages like Ancient Greek, but with some induced oddities and differences so as to avoid something repetitive or boring. For example, placing the rhotic consonant in a uvular-velar category to justify having a labialized version of it, as was done for the other back consonants. The intent was to have something that could feel relatively convincing as a naturalistic inventory while allowing some liberty. Finally, with the sound inventory selected, syllable structure and allophony are set up as something not overbearing that could serve as a basis for pursuing a highly phonaesthic work while still possibly challenging both the author and readers with pronouncing the language out loud if they want to. While what is and is not pleasing to the ear is tremendously subjective, work invested in the phonology of the language hopefully only adds to senses of naturalism and enjoyment for anyone who encounters \langname.\par
  The grammar, however, is far more tied to real world languages. The aspectual system of the verbs, which makes up a large portion of the language, are strongly influenced by Mayan languages with some personal challenges included. Furthermore, conjunctive and adverbial usage of verbs was something that wanted to be explored deeply. The nouns, while much smaller in the grammatical space than verbs, were likewise meant to be a small personal challenge. The system was modeled very loosely on Japanese with grammatical number being more present, case marking affixes covering a slightly different space, and classifiers having slightly widened usage.\par

\subsection{The Speakers and Their Culture}
  The speakers of \langname while all living in loose proximity to each other do not all live homogenous lifestyles. Many groups are semi-sedentary, not living in one spot year round, but rather circulating a large area based on resource availability with some areas having permanent dwellings that are abandonned for months at a time. Other times they live within lighter, more mobile camps with no intention of duration for later or longer use. Some groups still live fully nomadic lives without any permanent housing subsisting on a more hunting and foraging strategy.

\subsection{High Level Linguistic Synopsis}
  \langname is a fairly synthetic, agglutinative language with a six pure vowels, some in two lengths, and a large cache of consonants that work within a simple phonotactic system. It is topic-prominent with nominative-accusative alignment using Verb-Subject-Object word order with some traces of having recently been Subject-Verb-Object. Subject dropping is strongly preferred when context allows it, and, because of topic fronting, the subject may at times appear to come before the verb. Overall it prefers head-final word order. Lacking grammatical tense, it makes liberal use of adverbs, temporal specifiers and connectives, and temporal anchoring on perfective verbs to indicate when an event happened; it liberally uses verbs in various conjunctive forms and constructions to tie related events to established topic times.\par
