\section{Preface}
\langname is a reconstructed historical language from which the \branchnorth\comma \branchhigh\comma and \branchlow\space are known to descend from with the \branchremote\space being considered highly likely as well in addition to other smaller groupings. There is no known or expected direct evidence of this proto-language in any form aside from what was inherited in daughter languages and loanwords into other languages present during its lifetime. Current estimates place its existence and spread during the later portion of the neolithic after the development of fired pottery; contact with it and its daughter languages brought into other languages several concepts that seem to be novel innovations or cultural relics from its speakers such as certain forms of companion planting, particular forms of kaolin vessels, and some stories that become more interculturally common with their spread.\par
On the whole, \langname had a modest inflectional morphology that was augmented by its syntax. Currently a matter of dispute is whether the language marked four or five nominal cases via bound suffixes. Some seemed to have been optional with simple placement in a phrase or subphrase being sufficient; however the ergative argument was always strongly marked. The expansion of that case in some daughter languages causes some of the current confusion over the number of cases asserted in reconstruction.\par
At present, we present our work to the best of our skills and the materials available. Should further evidence or more convincing arguments be made, we may revise this future editions, but for now it is our earnest hope that this reconstruction of our history will provide insights to learners and researchers in other fields.
