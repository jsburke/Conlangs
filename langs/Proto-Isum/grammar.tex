\section{Grammatical Overview}
This quick introduction aims to provide high level traits of \langname quickly without requiring the reader to search through other parts of the text. It will include no examples in the language itself, and exceptions to these large picture characteristics will only be presented here if the exception constitutes something that would be encountered frequently.
\subsection{Ergative Alignment}
\langname is a dominantly Ergative language. Namely, the subject of intransitive verbs and the object of transitive verbs receive the same marking, but subjects of transitive verbs are marked differently. This applies to all nouns, pronouns, and nominal phrases in the language in general.
\subsubsection{Quirky Subjects}
A small collection of verbs require their subject to be marked with a case that is not the Ergative or Absolutive; it is most often the Dative case used in this way. Almost all of these verbs also have specific usages for the normal ergative and absolutive cases, but often with restricted senses. Many of these verbs are 'mental actions' such as 'to like', 'to think about', or 'to remember'; many speakers thus culturally regard some of these rather laborious endeavors as more spontaneous. Causees in causative construction are often marked as Dative as well.
\subsection{Word Order}
\langname seems to have strongly preferred \texttt{SOV} word order, but free order was possible. Different orderings would have highlighted or diminished certain arguments to the verb. It seems the only mandatory rule, which some present day daughter languages and most of their predecessors enforced, was that the subject appeared before the object. While it isn't ungrammatical to reverse this order per se, doing so was so avoided that it effectively acts as a rule.
\subsubsection{Verb Medial}
Some reconstructions prefer \texttt{SVO} instead. Indeed, this structure was not uncommon, and it seems to have been prefered in certain environments like causatives.
\subsection{Prepositional Preference}
While \langname features some cases that cover prepositional phrases, the proto-language seems to have had some internal momentum towards prepositional usage while reducing some cases. The only daughter languages that strongly preserved case or innovated on it seem to have done so under pressure from languages outside of this family.
\subsection{Dispreference for Relative Clauses}
Relative clauses were likely used by the speakers of the proto-language, but evidence seems to indicate that their usage was often avoided in favor of using adjectival nouns (like gerunds or participles) or modifiers through genetive constructions. 
