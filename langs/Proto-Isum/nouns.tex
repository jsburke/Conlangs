\section{Nouns}
Grammatically, nouns are a fairly simple lexical category. All nouns are marked for nominal case at all times without exception. There are a few suffixes indicating different kinds of the associative plural; however, their use is only mandatory on personal pronouns. Certain nouns are never marked with such plurality while others have a disposition towards them.  Finally, there is a wealth of derivational affixes that attach to nouns used for a variety of purposes. A short list of those with examples are provided in this section, while a more exhaustive list of such affixes is provided in the appendices.

\subsection{Overview of Case suffixes}
\begin{wrapfigure}{l}{0.30\textwidth}
  \begin{tabular}{|l|l|l|l|}
    \hline
    Affix                    & Case       & Notes  \\ \hline \hline
    \suffixtext{\varnothing} & Nominative &        \\ \hline
    \suffixtext{\acc}        & Accusative &        \\ \hline
    \suffixtext{\gen}        & Genetive   &        \\ \hline
    \suffixtext{\dat}        & Dative     &        \\ \hline
    \suffixtext{\loc}        & Locative   &        \\ \hline
  \end{tabular}
\end{wrapfigure}
There are five distinct cases used to indicate structure. The Accusative and the Genetive both cover a wide range of uses beyond the common understanding for the cases cross linguistically. The Genetive can effectively act as an attributive sort of case, while the Accusative as the default object case is frequently found in constructions where the Dative and Locative may have also been options. Notably, the Locative case is fairly weakly used as it has a lot of overlap with prepositional like phrases with less specificity in regards to both location and what it is modified. Often, the Locative simply indicates where a particular action had taken place.

\subsection{The Nominative: \textbf{\suffixtext{\varnothing}}}
The nominative case, which is unmarked, indicates subects of sentences and clauses, much like in other languages that feature it. Furthermore, it is found in disjoint usages as well, such as responses to questions that only require specifying what a thing is with a bare noun.

\subsection{The Accusative: \textbf{\suffixtext{\acc}}}
The accusative case, marked with the \suffixtext{\acc} suffix, indicates that the noun is the target of some kind of verb or action. It's most common use is to indicate the direct object of verbs.
  \subsubsection{Sound Changes}
  Much like the plural, when a noun with a final \phonemic{m} is marked as accusative, the final consonant is dropped, such that \langsample{gwiim} becomes \langsample{gwii\acc} rather than \langsample{gwiim\acc} which would imply a geminated \phonemic{m} at the conjunction of the word and suffix.
  \subsubsection{Prepositional Objects}
  Prepositions effectively being a category of verbs, the object of such prepositional verbs is in the accusative case. For example, \langsample{chiici thi asa\acc\gen} is \transsample{the bird under the tree}, where \langsample{asa}, tree, is marked accusative. The subsequent genetive marker indicates that \langsample{chiici}, which comes before it, is modified.

\subsection{The Genetive: \textbf{-\gen}}
Genetive marking is used to indicate some form of attribution, such as possession and nominal or phrasal modification of nouns. Genetive nouns follow the noun or noun phrase that they modify. Thus, \langsample{gwiim sulaa\gen}, is \transsample{the person's house}. Reversing it to \langsample{sulaa\gen\space gwiim} was regarded as incorrect and, while understandable would feel very odd.
  \subsubsection{Sound Changes}
  When a word ends with \phonemic{n}, much like the accusative and plural with \phonemic{m}, the \phonemic{n} would be dropped and the affix added. For example, \langsample{can} would become \langsample{ca\gen}.
  \subsubsection{Nominal Augmentation}
  Nouns on their own, cannot modify other nouns as in English since there is no implied zero-derivation to a verbal form. For example, \langsample{byna mui\gen} is literally \texttt{milk of the cow}, but idiomatically is simply \transsample{cow milk}.
  \subsubsection{Relative Clauses}
  Relative clauses hinge on the use of the genetive case marker. While covered more in its own section, relative clauses follow the noun they modify with, optionally, one of the demonstrative pronouns after it marked genetive. For example, \langsample{Sulaa cyri chiici\acc\space \proxdem\gen} is translated as \transsample{The man who sees the bird}. As will be seen, the demonstrative could optionally be dropped.

\subsection{The Dative: \textbf{-\dat}}
The Dative case indicates the indirect object, motion towards something, and was used to mark subjects for some verbs and for some lower volition subjects as well. The affix, \suffixtext{\dat}, is notable for having quite a bit of volatility in its own sound. It was never used to indicate any sort of benefactive use.
  \subsubsection{Quirky Subjects}
  A collection of particular verbs expected Dative marked nouns as their subject; when these ones have direct objects they are always in the accusative case. Many of these verbs have to do with mental activities like thinking or feeling affection. Culturally, this seems to have lead some people to feel that aspects of these things were more spontaneous to the human mind than meditated, which gave rise to a few idioms and collocations which manifest in certain ways in the daughter languages
  \subsubsection{Absence of Benefactive Usage}
  Very notably, \langname did not make use of the Dative to indicate those who benefitted from an action or state, at least directly. Giving food or affection surely benefits the person, but this was not something overtly indicated by the Dative Case. The Proto-language and almost all of its daughter languages up to the present day use subordination of some kind for benefactive indicators. Those that presently do grammatically use the Dative in a benefactive sense seem to have innovated it through contact with neighboring languages that do.

\subsection{The Locative: \texttt{-\loc}}
The Locative case indicates where something occurs. The sense entailed by the Locative is that of "in, on, or at". For example, \textbf{gwiim\loc}, in the house. However, the locative is not used for purely copular forms, such as \texttt{The birds are in the tree} would be rendered as \textbf{zenyi qhiiqimu asa\acc} as opposed to using the possibly expected \textbf{asa\loc}
