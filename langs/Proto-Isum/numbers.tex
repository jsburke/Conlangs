\section{Numbers}
\langname had a fairly direct number system. It was decimal with a quinary sub-base. The section essentially only presents basic numbers, how to combine and build further numbers using the language's simple pattern, and how to create ordinal numbers from cardinal ones. Of note, since the language was particularly old, there was no concept of zero, and many of the higher order numbers were interchangeable with meanings like "many" or "abundant" and are thus also verbs.\par
\begin{tabular}{|l|l|l|l|}
  \hline
  Number  & Cardinal  & Ordinal            & Notes              \\ \hline \hline
  1       & \numone   &                    &                    \\ \hline
  2       & \numtwo   &                    &                    \\ \hline
  3       & \numthree &                    &                    \\ \hline
  4       & \numfour  &                    &                    \\ \hline
  5       & \numfive  &                    & Sub-base           \\ \hline
  6       & \numsix   &                    & 5 + 1              \\ \hline
  7       & \numseven &                    & 5 + 2              \\ \hline
  8       & \numeight &                    & 5 + 3              \\ \hline
  9       & \numnine  &                    & 5 + 4              \\ \hline
  10      & \numten   &                    & Major Base         \\ \hline
\end{tabular}

