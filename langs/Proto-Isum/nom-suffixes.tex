\section{Nominal Suffixes}
\langname used a handful of nominal suffixes to indicate mainly case, and more rarely number and disparagement. Any noun or pronoun could be marked for case. Human nouns and pronouns had mandatory plural marking. And all nouns, but never pronouns, could be marked with a pejorative suffix sort of meaning "crappy" or "stupid".\par

\subsection{The Plural: \texttt{-\plural}}
\langname had a plural marker \texttt{-\plural}; however, despite its wide spread usage in some daughter languages, plural marking was far less common in the past. Scholastic consensus consistently holds that plural marking has always been mandatory for the first and second person pronouns. Third person pronouns, being the same as the demonstrative pronouns, it would have remained mandatory for human references, but for others it seems unlikely given both plural marking in daughter languages and pronominal systems in them. Likewise, nouns that describe humans of any sort had mandatory plural marking. Some animals may have received plural marking, but it is far harder to discern if it was consistent.\par
Plural marking, when used was always most closely bound to the noun, coming before both case or disparagement suffixes.\par

\subsection{The Ergative (\texttt{-\erg/-\ergold}) and Absolutive Cases (\texttt{-\varnothing})}
The Ergative and Absolutive cases are used to indicate the most common nominal arguments to verbs and in utterances in general. They play a role akin to the Nominative and Accusative in many world languages; however, what they mark is dependent on both the transitivity of the verb. The Absolutive case, which is unmarked, indicates the object of transitive verbs and the subject of intransitive verbs regardless of volition. The Ergative case, most often marked with the \texttt{-\erg} suffix, indicates the subject of transitive verbs, and it has some other syntactic usages as well. The first and second personal pronouns along with a small collection of proper nouns, which seem sporadic in nature, are marked with the \texttt{-\ergold} suffix instead; this different suffix is not used for any extra syntactic roles, but its sound seems more conservative so it is thought it simply fossilized on these particular nominals before the particle as a whole shifted to \texttt{-\erg}, or it could be the remnant of one dialect pushing on another. It is hard to discern fully.
  \subsubsection{Historical Relation to the Genetive}
  As will be read in a later section, the Genetive Case is marked with the \texttt{-\gen} suffix. Given the similarity in sound both to the general and conservative Ergative suffixes and the tendency of \langname to "own" participle verbs in various situations, it feels overwhelmingly likely that the Ergative Case has it's origin in particular usages of the Genetive Case. This may have happened at a time also when an unattested Ablative suffix was fusing with the Genetive seeing how Ablative phrases often manifest with Ergative marking.
  \subsubsection{Ergative Subject}
  The Ergative Case has its most obvious use marking the subject of transitive verbs. Here, we provide a rather plain example:
  \example{\fstppn\ergold\space raam nyawi}{1SG.ERG food(ABS) eat}{I'm eating food}
  \subsubsection{Instrumental Usage of the Ergative}
  The Ergative suffix is also used to mark nouns for instrumental usage. Most notably, this happens with the preposition \instruprep\space which simply means \textit{with} in cases of use but not for accompanyment; however the usage of that preposition seems to have not been mandatory. Often placing an ergative marked noun after the verb was enough.
  \subsubsection{Ablative Usage of the Ergatives}
  \subsubsection{Ergative Marking for Adverbials}

\subsection{The Genetive: \texttt{-\gen}}
Genetive marking was used to show possesion by one noun over another, or to mark some other close relationship or association or origin. It is indicated with the \texttt{-\gen} suffix on the possesee, whic always follows the noun or noun phrase being owned or modified. There is no notion of alienability in this way.
  \subsubsection{Agentive Usage of the Genetive with Passive Verbs}
  In Passive constructions, agents of the action can be optionally indicated by a noun marked for the Genetive Case following the verb. Since genetive ordering always has it following the possesum and, internal to the language, since it feels like the passivized verbal phrase is owned by the noun, grammatical pressures make promoting agents of the passive to before the verb or verb phrase feel ungrammatical.

\subsection{The Dative: \texttt{-\dat}}
The Dative Case was marked with the \texttt{-\dat} suffix, and it was used to indicate motion towards something or note indirect objects. Promoting a Dative noun to the subject role via passivization of the verb was ungrammatical. Some daughter languages would innovate a genetive use out of the dative, but this seems to not be present in the Proto-language.
  \subsubsection{Quirky Subjects}
  A collection of particular verbs expected Dative marked nouns as their subject; when these ones have direct objects they are always in the absolutive case. Many of these verbs have to do with mental activities like thinking or feeling affection. Culturally, this seems to have lead some people to feel that aspects of these things were more spontaneous to the human mind than meditated, which gave rise to a few idioms and collocations which manifest in certain ways in the daughter languages
  \subsubsection{Absence of Benefactive Usage}
  Very notably, \langname did not make use of the Dative to indicate those who benefitted from an action or state, at least directly. Giving food or affection surely benefits the person, but this was not something overtly indicated by the Dative Case. The Proto-language and almost all of its daughter languages up to the present day use subordination of some kind for benefactive indicators. Those that presently do grammatically use the Dative in a benefactive sense seem to have innovated it through contact with neighboring languages that do.


\subsection{The Locative: \texttt{-\loc}}
\subsection{The Oblique: \texttt{-\obl}}
No clue, quirky subjects I guess too
