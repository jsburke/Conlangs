\section{Classifiers}
Classifiers and measure words are used to specify nouns when bening counted, ordered, or specified in certain particular ways. Classifiers are completely non-inflectional; however some are highly subject to environmental sound changes. Phonetically, classifiers in \langname\space are all one syllable with a short vowel, and none have aspirate consonants. Unlike many other languages with classifiers, the plural marker may co-occur with explicitly numeral classified nouns. This is largely mandatory for humans, but strongly avoided otherwise. Finally, some classifiers have an implicit number, such as for things that come in pairs or that refer to groups.

\subsection{Synactic Positioning} 
The ordering of words in a noun phrase that uses a classifier is \texttt{Verbal Adjective - Noun - Demonstrative - Number - Classifier}. Of note here, using a demonstrative without a classifier can change what is understood. Furthermore, if elements of this construction can understood implicitly, it may be dropped with the meaning preserved, sometimes even including the noun. 

\subsection{Classifier Selections}
Each noun will have a selection of possible classifiers that can be used with it. First, there is a completely generic classifier, \langsample{\clfgen}. This classifier may be used with any noun at any time; however, it does feel a bit less refined and overuse of it will feel monotonous and lazy. Beyond that, there each noun is also associated with a more particular classifier based on its perceived characteristics. For example, \langsample{\clfrounddull} is the classifier for small round dull things, such as \langsample{can}, rock, and \langsample{deklu}, charcoal. A smaller category of nouns which refer to things that come in pairs, especially body parts, or natural groups, such as clusters of certain kinds of fruit. In these cases, there is one classifier for individual things apart from the assumed group, such as \langsample{\clfbodysing} for one eye whereas \langsample{\clfbodyjoin} is for the pair or eyes. Finally, there is a group of classifiers used for common but not inherent collections of nouns, such as \langsample{\clfsack} for a sack or skin full of something like meat or clay.

\subsection{Enumeration}
Maybe the most prominent usage of nominal classifiers will be specifying the exact number of a noun. Following the syntax section above, we can follow a simple pattern to show specific numbers. For example, \langsample{can \numthree\space \clfrounddull} is directly \transsample{three rocks}. Likewise, the same can be done with the generic classifier in \langsample{can \numthree\space \clfgen}.

\subsection{Ordinals}
third cat

\subsection{Definiteness}
this cat, those 3 cats, which cat

\subsection{Quantification}
Some cats, every cat, no cat

\subsection{List of Classifiers}
\begin{tabular}{|l|l|l|l|}
  \hline
  Classifier    & Main Usages                                & Sound Changes  & Notes \\ \hline \hline
  \clfgen       & Generic Classifier                         &                &       \\ \hline
  \clfstick     & Sticks, long thin stiff objects            &                &       \\ \hline
  \clfrounddull & rocks, charcoal lumps, small round things  &                &       \\ \hline
  \clfbodysing  & One of a paired body part like hands       &                &       \\ \hline
  \clfbodyjoin  & Body parts as naturally paired or gathered &                &       \\ \hline
  \clfsack      & Inaninmate things collected in sacks       &                &       \\ \hline
\end{tabular}

