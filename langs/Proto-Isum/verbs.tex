\section{Verbs}
\langname morphologically splits all verbs into one of two categories, transitive or intransitive. Transitive verbs in dictionary form end with \suffixtext{\transv} while intransitive verbs with \suffixtext{\intransv}, barring a few irregular ones which will be enumerated separately. Each verb is lexically inferred to exist in one of three default temporal categories: instantaneous, delimited, and essential. There is no grammatical tense, but an abundance of aspectual categories that each verb may be marked with that can. Understanding of a given verbs usage primarily relies on the interaction of the temporal category and aspect. For example, an instantaneous verb like \textbf{iriot}, to squeak, marked with the continuous aspect becoming \textbf{iri\cont} means "is squeaking" or "was squeaking". With no overt grammatical tense, disambiguating when an event occurs is moreso in the domain of adverbials indicating time, sequential coordinators like "before" and "while", and anchoring stories or topics to verbs in perfective aspect.


\subsection{Passive Voice}

\subsection{Causative}


\subsection{Adjectival Verbs}
Adjectives are not a distinct lexical category, so verbs cover the majority of adjective like activities in \langname. The most common verbs to act in this way are stative intransitive verbs, like \textbf{kajot}, \textit{big} or \textit{to be big} or \textbf{thaarot}, \textit{good} or \textit{to be good}. The verb in this usage will come before the noun being modified in the noun phrase, for example \textbf{Cyri dii kajot etu\acc\space awa\gen}, \textit{I saw your big father}, where \textbf{kajot} modifies \textbf{etu} by coming before it. Of note, simple subject predicate sentences with intransitive stative verbs would still idiomatically be \texttt{X is Y} in translation, such as \textbf{kajot saktu} is better understood as \textit{The sky is big} rahter than \textit{big sky}, which would be better as a noun phrase embedded in some other utterance.

\subsection{Prepositional Verbs}
Prepositions being an absent category, when nominal case does not cover a particular meaning, verbs cover this need. All prepositional use verbs are transitive stative verbs. For example \textbf{thi etu diinye asamin} is the predicate usage of the verb for \textit{to be under}, so this is translates to \textit{My father is under the tree}. However, when used in a fashion like a prepositional phrase the structure is \texttt{NOUN PREP OBJECT\acc\space DEMONSTRATIVE\gen}; this is exactly using a relative clause containing the prepositional phrase. The demonstrative used related where the modified noun is perceived to be in relation to the speaker, thus something close to the speaker would expect \textbf{\proxdem\gen}. Some daughter languages imply that the demostrative was optional as they use a gapping strategy. So, \textbf{Cyri dii qhiiqi\acc\space awa\gen\space ybi gaaru\acc\space \distdem\gen} is \textit{I saw your bird near the pond}.
