\section{Pronouns}
Personal pronouns were rather simple, with the third person pronouns normally being the proximate demonstrative pronouns, though very distant non-moving or conceptual things could be refered to by the other demonstratives. As with the rest of the language, only humans are marked for plurality. This means that multiple people noted via pronoun could be marked plural, but multiple trees or rocks would not ever be plural marked either on the noun or pronouns refering to them. The following sections layout what the pronouns are, though there are few wrinkles in this arena.\par

\subsection{First and Second Person Personal Pronouns}
  \begin{center}
  \begin{tabular}{|c|c|c|c|c|}
    \hline
                  & First Singular & Second Singular & First Plural          & Second Plural        \\ \hline \hline
      Nominative  & \fstppn        & \scdppn         & \fstppn\plural        & \scdppn\plural         \\ \hline 
      Accussative & \fstppn\acc    & \scdppn\acc     & \fstppn\plural\acc    & \scdppn\plural\acc     \\ \hline 
      Genetive    & \fstppn\gen    & \scdppn\gen     & \fstppn\plural\gen    & \scdppn\plural\gen     \\ \hline 
      Dative      & \fstppn\dat    & \scdppn\dat     & \fstppn\plural\dat    & \scdppn\plural\dat     \\ \hline 
      Locative    & \fstppn\loc    & \scdppn\loc     & \fstppn\plural\loc    & \scdppn\plural\loc     \\ \hline 
  \end{tabular}
  \end{center}
\par\par
In this section, we only list the First and Second personal pronouns since the 3rd is the same as the demonstratives, and notes on its use will come in that section. These two pronouns are notably the only ones reconstructed as having mandatory plural marking with \texttt{-\plural}. Vestiges of this appear in all daughter languages, even ones that did not extend this plural marker much beyond the space present in the proto-language here. It's notably where some of the irregularity in daughter languages can find its source.\par

\subsection{Demonstrative Pronouns}
  \begin{center}
  \begin{tabular}{|c|c|c|c|c|c|c|}
    \hline
                  & Proximal Singular & Medial Singular & Distal Singular & Proximal Plural        & Medial Plural        & Distal Plural \\ \hline \hline
      Nominative  & \proxdem          & \meddem         & \distdem        & \proxdem\plural        & \meddem\plural       & \distdem\plural      \\ \hline 
      Accussative & \proxdem\acc      & \meddem\acc     & \distdem\acc    & \proxdem\plural\acc    & \meddem\plural\acc   & \distdem\plural\acc  \\ \hline 
      Genetive    & \proxdem\gen      & \meddem\gen     & \distdem\gen    & \proxdem\plural\gen    & \meddem\plural\gen   & \distdem\plural\gen  \\ \hline 
      Dative      & \proxdem\dat      & \meddem\dat     & \distdem\dat    & \proxdem\plural\dat    & \meddem\plural\dat   & \distdem\plural\dat  \\ \hline 
      *Locative   & \proxdem\loc      & \meddem\loc     & \distdem\loc    & \proxdem\plural\loc    & \meddem\plural\loc   & \distdem\plural\loc  \\ \hline 
  \end{tabular}
  \end{center}
\par\par
Demonstrative pronouns, such as "this" and "that" in English, is split into three spatial categories. The proximal, \textbf{\proxdem}, refers to objects close to the speaker or events relatively close in time, both past and future. The medial, \textbf{\meddem}, refers to things closer to the listener or near term events, such as within the past day or so or expected in the next few days. Finally, the distal, \textbf{\distdem}, indicates things that are far from both the speaker and listener or events deemed a bit more remote, such as prior or upcoming weeks. Of note, locative demonstratives are relatively rare because place words were prefered; in fact, in some daughter languages where case remained fairly in tact, fused locative demonstratives are seemingly absent. Furthermore, plural demonstratives would be less used given the predisposition for assumed plurals.

\subsubsection{Third Person Usage of Demonstrative Pronouns}
The demonstrative pronouns \langsample{\proxdem} and \langsample{\distdem} were used where most languages have a third person pronoun. \langsample{\meddem} was not used in this way. When used in reference to other people, groups, or beings regarded with some intellect, mandatory pluralization was present just like in the first and second person pronouns. The choice of proximate or distal pronoun was governed by situational and conversational context. When no other third person referent was easily in mind, the choice was simply based on presence; people near by would be proximate, while those out of sight or dead are distal. However, if multiple third person actors are present in a story or conversation, the use of the proximate demonstrative indicates a referent who was more recently in charge of action while distal more often was for those on the receiving end of said actions. For example, in \langsample{Cyri etu ymaa\acc asa\loc. Bulot\cont \distdem} is \transsample{At the tree, dad saw mom. She was walking around}. Here we know it has to be mom walking around because of the use of the distal demonstrative pronoun.
