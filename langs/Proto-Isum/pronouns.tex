\section{Pronouns}
Personal pronouns were rather simple, with the third person pronouns normally being the proximate demonstrative pronouns, though very distant non-moving or conceptual things could be refered to by the other demonstratives. As with the rest of the language, only humans are marked for plurality. This means that multiple people noted via pronoun could be marked plural, but multiple trees or rocks would not ever be plural marked either on the noun or pronouns refering to them. The following sections layout what the pronouns are, though there are few wrinkles in this arena.\par
\subsection{First and Second Person Personal Pronouns}
\textbf{N.B.:} John this section and its table will need to be much more fleshed out for case suffix interactions. Return to this soon!!
  \begin{center}
  \begin{tabular}{|c|c|c|}
    \hline
                  & Singular & Plural         \\ \hline \hline
      First       & \fstppn  & \fstppn\plural \\ \hline 
      Second      & \scdppn  & \scdppn\plural \\ \hline 
  \end{tabular}
  \end{center}
\par\par
In this section, we only list the First and Second personal pronouns since the 3rd is the same as the demonstratives, and notes on its use will come in that section. These two pronouns are notably the only ones reconstructed as having mandatory plural marking with \texttt{-\plural}. Vestiges of this appear in all daughter languages, even ones that did not extend this plural marker much beyond the space present in the proto-language here. It's notably where some of the irregularity in daughter languages can find its source.\par
