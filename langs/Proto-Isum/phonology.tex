\section{Phonology}
The present reconstruction of \langname presents a phonology with a rather balanced and fairly full set of phonemes. There are few gaps within its patterns, and points that might otherwise seem unusual seem to fit some kind of internal pattern. The realizations in the daughter languages is quite varied. In the documentation provided below, the phonemic elements are presented plainly while the Romanization used widely is in angle brackets. 
  \subsection{Vowels}
     \begin{wrapfigure}{r}{0.55\textwidth}
       \begin{tabular}{|l|l|l|l|}
         \hline
                 & Front              & Central            & Back               \\ \hline \hline
         High    & i, \tlongi <i, ii> & \tshorty* <y>      & u, \tlongu <u, uu> \\ \hline
         Mid     & \tshorte <e>       &                    & \tshorto <o>       \\ \hline
         Low     &                    & a, \tlonga <a, aa> &                    \\ \hline
       \end{tabular}
     \end{wrapfigure}
     \langname features at least 5 vowel qualities with three at two lengths with no quality differences between long and short. It is notable that in multisyllabic words, at most one long vowel is present; unsurprisingly it is always realized as the stressed syllable. Long vowels seem to have been more resistant to synchronic changes. Some scholars contest the inclusion of /\tshorty/ in the reconstruction, but we believe that this inclusion is sensible based on reflexes found in the \branchremote\space languages. When they are not considered in this equation, its exclusion seems sensible, but certain oddities in how the \branchremote\space presents vowel reflexes in stressed syllables demands its inclusion, in our opinion, for the most thorough analysis.\par
  \subsection{Consonants}
    \begin{center}
    \begin{tabular}{|l|l|l|l|l|}
      \hline
                    & Labial      & Alveolar                &  Palatal            & Velar - Uvular                  \\ \hline \hline
        Plain       & p <p>       & t, \tplainc <t, c>      &  \tplainq <q>       & k, \tplainkw  <k, kw>           \\ \hline 
        Voiced      & b <b>       & d, \tvoicedz <d, z>     &                     & g, \tvoicedgw <g, gw>           \\ \hline 
        Aspirate    & \taspp <ph> & \taspt, \taspc <th, ch> &  \taspq <qh>        & \taspk, \taspkw <kh, khw>       \\ \hline 
        Nasal       & m <m>       & n <n>                   &  \tpaln <ny>        &                                 \\ \hline 
        Fricative   &             & s <s>                   &  \tshlig <x>        & x, \tplainxw <h, hw>            \\ \hline 
        Approximant &             & l <l>                   &  j, \tpall <j, ly>  & w, \tbackr, \tbackrw <w, r, rw> \\ \hline
    \end{tabular}
    \end{center}
  \par\par
  \langname is reconstructred as having a three types of stops at four points of articulation using two methods of release (full stop and affricate). The velar and uvular consonants present with both rounded and unrounded versions. The single rhotic /\tbackr/ follows this pattern as well. Notably not present in this reconstruction is both /f/ and /\textipa{\t{dZ}}/, which seem to be holes in the otherwise strongly filled consonant space. Some scholars squabble over the inclusion of /\tpall/ prefering instead to split it among /l/, /j/, and /\tpaln/ under various conditions. We have chosen against that since we feel it distills into daughter languages neatly despite being a comparatively rare phoneme both within the lexicon and also within expected discourse usage.\par
  \subsection{Syllable Structre}
  Reconstructions consistently agree on the syllable structure being \texttt{(C)V(m/n/p/t/k)} where coda stops were unreleased. The nucleus may be either short or long, but not a diphthong. Various aspects of this structure and how it interacts with perceived stress and prosody led to the wide varying realizations found in daughter languages.\par
  \subsection{Allophony}
  Every language has some variety in how underlying phonemes are realized when placed in different environments, and \langname was surely no different from that. Some of the synchronic changes that are expected to have existed are evidenced in some way in daughter languages. Please note, this is restricted to expected changes triggered by environment by the speakers of \langname and not changes witnessed in its various daughter languages.\par
    \subsubsection{Velarizing /n/}
    /n/ almost definitely presented as [\textipa{N}] before velar stops and fricatives and at the end of words. However, change to /n/ seems unlikely before the uvular rhotics.
    \subsubsection{Intervocalic /w/}
    /w/ became [v] in all intervocalic word medial positions. No daughter languages present any convincing evidence that it remained a semivowel in this position.
    \subsubsection{Intervocalic Fortified /j/}
    Based on reflexes in most daughter languages, it seems highly likely that when /j/ followed a stressed vowel in word medial intervocalic positions, it was realized as [zj].
    \subsubsection{Interruptive Glottal Stop}
    While not present as a distinct phoneme, it is expected that the glottal stop, [\textipa{P}], was inserted to break vowel-vowel sequences across word boundaries.
  \subsection{Final Notes on Romanization}
  The above sections enumerate everything that is needed for the Romanization; however some digraphs like <kh> and even <khw> could be confusing without disambiiguation. Intervocalic aspirate stops could be confused with clusters of coda stops and the velar fricative. For this reason, when [kx] or other clusters like this occur, the coda stop will be separated with and apostrophe <'>. Thus <akha> is [a\taspk a] but <ak'ha> is [akxa].
  \subsection{Stress}
  \langname did not make use of contrastive stress, that is stress patterns to distinguish morphemes, but for multisyllabic words in particular stress was present. Many frequently used syntactic particles that were not affixed to main words, and probably many common use words, often acted as unstressed when juxtaposed with more lexically prominent words. Stress within a word was governed seemingly by a few simple rules:\par
  \begin{itemize}
    \item Monosyllabic words, in isolation, have no stress pattern
    \item Long vowels in multisyllabic words are always stressed
    \item If there are no long vowels, the syllable closeset to the end of a word before a consonant cluster not including fricatives or approximnats had stress
    \begin{itemize}
      \item\textbf{[Example]}: \texttt{saktu} will have stress on the first syllable}
    \end{itemize}
    \item If there are no clusters, and the final syllable is open, stress is penultimate
    \begin{itemize}
      \item\textbf{[Example]}: \texttt{iwa} will have stress on the first syllable
    \end{itemize}
    \item Finally, if there are no clusters, and the final syllable is closed, stress is ultimate
    \begin{itemize}
      \item\textbf{[Example]}: \texttt{lyagwyn} has stress on the final syllable
      \item\textbf{[Example]}: \texttt{khakpet} has stress on the penultimate syllable
      \item\textbf{[Example]}: \texttt{iixyt} has stress on the penultimate syllable
    \end{itemize}
  \end{itemize}
