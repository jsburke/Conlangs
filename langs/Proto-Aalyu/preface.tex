\section{Preface}
The \langname language is intended as a Protolanguage for a world building project. The origin of this language group comes from a people who are partially agricultural and partially pastoral in an era before the pottery neolithic. The people originate from near a river system fairly far inland. In the warmer months, they hunt and graze, while accumulating some stock, especially fish, for the winters. During the winters, the groups coallesce into semi-permanent camps and rely on stored fish, tubers, and mildly fermented goods. They also continue to fish some.

\langname derives much of its phonological system from Australian languages, but with a moraic system. It is strongly head marking and makes extensive use of verb derivation to express complex meanings. Nouns have an unmarked implicit hierarchy based on animacy, which is heavily influenced by religious perceptions.
