\section{Nouns}
Nouns in \langname are clearly an inflectional category of words, however the degree of morphology attached to nouns is fairly light. All nouns have an implicit animacy that fits into a loose hierarchy which has some minor effects on how nouns may be inflected. As will be seen, this animacy hierarchy has some importance in the verbal morphology as well. Furthermore, third person nouns are marked for obviation for various reasons ranging from sentence level distinctions, to thematic roles, to the general flow of discourse.

\subsection{The Animacy Hierarchy}
All nouns in \langname have an inherent animacy. This goes a little beyond a simple animacy dichotomy instead relying a bit more on cultural expectations for a given things perceived agency. In this way, while something like a plant is seen as animate, it is viewed as having less agency than an animal. These perceptions, as will be seen, play a large role in verbal morphology.
\vspace{3mm}
  \begin{wrapfigure}{l}{0.75\textwidth}
    \begin{tabular}{|l|l|}
      \hline
      Class             & Examples                                    \\ \hline \hline
      Autonomous        & Humans, Spirits                             \\
      Highly Autonomous & Children, Pets                              \\  % predates domesticated livestock
      Higher Thinking   & Wolves, Birds, Whales and Orcas, Predators  \\
      Instinctual       & Alligators, Rabbits, Mice                   \\
      Reactive          & Small Lizards, Insects                      \\
      Alive             & Plants, Fungi, Moss, Fermentables           \\
      Inorganics        & Minerals, Wood, Cloth, Water, Food, Fire    \\
      Abstracts         & Sounds, Plans, Stories                      \\ \hline
    \end{tabular}
  \end{wrapfigure}
  This table gives a rough illustration of how agency typically functions in \langname. That said, certain things may end up higher or lower depending on general or particular perceptions of agency. For example, if a spirit is felt to be motivating something, be it a sound or snowfall, it may be given a higher animacy, or, if an animal is acting very unusually it may be seen as having a lower one. It is far more common for lower animacy to be perceived as higher than the other way around. 

\subsection{Common Nominal Particles and Patterns}
There a handful of general use particles and patterns in \langname. With the exception of a couple, they always join at the end on the noun stem, and several may stack on top of each other.
  \subsubsection{True Plurals}
  Nouns may be marked for the plural following a variety of rules. For the human and spiritual agents, plural marking is mandatory, especially on pronouns. For anything in the reactive or higher degrees of agency, plural marking is common if not preferential; however sometimes it may be reasonably dropped. For example, if the speaker introduced a pack of wolves or a school of fish as a matter of discussion, the plural marking on them may be implied after the first mention. Plural marking is rather rare for things like minerals and mass nouns like water. It is not uncommon to hear a number modifying an apparently singular inorganic noun.\par
  Regardless, the -\textbf{l(u)} particle is used to pluralize nouns. If this is placed on an open syllable, it surfaces as -\textbf{l}, but for syllables with coda consonants, it becomes -\textbf{lu}.

\subsection{Personal Pronouns}
  \begin{wrapfigure}{l}{0.55\textwidth}
    \begin{tabular}{|l|l|l|l|}
      \hline
              & Front              & Central            & Back               \\ \hline \hline
      High    & i, \tlongi <i, ii> &                    & u, \tlongu <u, uu> \\
      Low     &                    & a, \tlonga <a, aa> &                    \\ \hline
    \end{tabular}
  \end{wrapfigure}
