\section{Nouns}
Nouns in \langname are clearly an inflectional category of words, however the degree of morphology attached to nouns is fairly light. All nouns have an implicit animacy that fits into a loose hierarchy which has some minor effects on how nouns may be inflected. As will be seen, this animacy hierarchy has some importance in the verbal morphology as well. Furthermore, third person nouns are marked for obviation for various reasons ranging from sentence level distinctions, to thematic roles, to the general flow of discourse.

\subsection{The Animacy Hierarchy}
\langname displays a three level animacy hierarchy. Though mostly the same throughout, the different levels are treated differently in minor ways through the language, and the animacy system plays a significant role in verbal morphology. The 3 classes distinguish percieved inanimacy, animacy, and sentience, they are listed here with some examples.
\vertspace
  \begin{wrapfigure}{l}{0.65\textwidth}
    \begin{tabular}{|l|l|}
      \hline
      Class           & Examples                               \\ \hline \hline
      Agentive        & Humans, Spirits, dogs, horses, hawks   \\
      Animate         & Small birds, reptiles, insects, plants \\  % predates domesticated livestock
      Inanimate       & Rocks, water, tools, housing, ideas    \\ \hline
    \end{tabular}
  \end{wrapfigure}
  This table gives a rough illustration of how agency typically functions in \langname. That said, certain things may end up higher or lower depending on general or particular perceptions of agency. For example, if a reptile of some sort is seen as threatening or scheming it may be sporadically elevated to the agentive class. Lowering animacy can happen, but is exceedingly rare. Though one might think it would be useful for a rude insult, that too is very very rare. The culture of insults works rather differently.

\subsection{Common Nominal Particles and Patterns}
There a handful of general use particles and patterns in \langname. With the exception of a couple, they always join at the end on the noun stem, and several may stack on top of each other.

  \subsection{True Plurals with \suffixtext{\agtconsplural}, \suffixtext{\animplural}, and \suffixtext{\inanplural}}
  Plurals can be denoted on all nouns, but are only mandatory on agentive nouns and personal pronouns. The plural suffixes are \suffixtext{\agtplural}, \suffixtext{\animplural}, and \suffixtext{\inanplural} for the agentive, animate, and inanimate noun classes respectively. For agentive nouns, this surfaces as \suffixtext{\agtvowelplural} after vowels and as \suffixtext{\agtconsplural} after consonants. Some simple examples follow:

  \example{raama\agtvowelplural}{spirit.PL}{spirits}

  \subsection{Associative Plurals via Partial Reduplication}
  Associative Plurals are common in \langname, and are used to indicate groups or collections of nouns that are the same or similar. For all noun classes, the associative plural is denoted by reduplicating the first syllable, with some rules based on the structure of the initial syllable. For syllables with an initial consonant, only the initial consonant-vowel sequence is reduplicated with coda consonants ignored. For words with initial vowels, if the vowel is short it will become long. However, if a word has an initial long vowel, reduplication is not possible. This process is never applied to given names, and there is an aversion towards its use for very specific individuals.
  Certain consonants also mutate in the reduplication process. In particular, all the lenis stops lenite to approximants or laterals. This is enumerated in the table below:

  \vertspace
  \begin{tabular}{|l|l|l|l|}
    \hline
    Consonant   & Mutated    & Base Word        & Reduplicated Example         \\ \hline \hline
    \textbf{b}  & \textbf{w} & Baani - Bird     & Baawaani - A flock           \\
    \textbf{d}  & \textbf{l} & Dukya - Mountain & Dulukya  - Mountain Range    \\
    \textbf{dh} & \textbf{l} & Dhan  - A Fruit  & Dhalan   - A Bunch of Fruits \\
    \textbf{gy} & \textbf{j} & Gyaal - A Bee    & Gyaajaal - A Swarm of Bees   \\
    \textbf{g}  & \textbf{w} & Gara  - Sand     & Gawara   - Desert            \\ \hline
  \end{tabular}
  
  \vertspace
  \example{Yama -- Yayama}{tuber -- RED.tuber}{tuber -- a bunch of tubers}

  \subsection{Obviation with \suffixtext{\agtconsobv} and \suffixtext{\inanobv}}
  Obviation is a marking system that is used to distinguish thrid person nouns that might otherwise be confused.  For simple sentences, this allows for clarity in who effects the verb on the other and the degree of volition for lexcially reciprocal verbs. At a discourse level, it keeps competing topics clear. Both agentive and animate verbs share a suffix for indicating the obviative, \suffixtext{\agtobv}, which is \suffixtext{\agtvowelobv} after vowels and \suffixtext{\agtconsobv} after consonants. Inanimate nouns are always marked with \suffixtext{\inanobv}.
  It should be noted, obviation is only marked when the two nouns are of the same animacy or in certain cases where a lower animacy noun exerts on a higher one. In the general case, if two third person nouns of different animacy are present, the higher animacy is assumed to be proximate and the lower obviative.

  \subsection{Nominal Possession}

    \subsubsection{The \infixtext{\possessive} infix}
    Possession in \langname is marked via the infix \infixtext{\possessive}. This does not vary based on the animacy of the constituent nouns. This infix is however not used when possessive pronouns may fill the role. Furthermore, when two third person nouns, including pronouns, are the constituents of a possessive noun phrase, the possessed noun will always be marked as obviative

    \example{Aamu ini bawa\agtvowelobv}{Mom POSS dog.OBV}{Mom's dog}

    \subsubsection{Deferential 1st and 2nd Person Possessives}
    Possession by 1st and 2nd person actors can be indicated through deferential patterns using Bound Adjectives, covered in a later section, and often demonstratives, also covered in a later section. These patterns are not strict possessives because they can be used in a more generic literal sense as well. Essentially, 1st person possession can be politely hinted at by disparaging a nominal with bound adjectives, namely the \prefixtext{\boundbad} prefix mentioned in the Bound Adjectives section, while elevating it with the \prefixtext{\boundtrue} prefix can politely indicate 2nd person possession. Frequently, the 1st person disparagement will involve also affixing the proximal demonstrative, and 2nd person one of the distal markers, often the mediodistal marker. Demonstratives are also covered in a later section. 3rd person possession in this way, is not evident in the protolanguage.
    There are limitations to what kinds of nouns can be possessed with this hinting pattern. While most will be of the agentive class, animacy is not the dividing line. Normally, family members, close friends, and other elements closely tied to heartfelt relationships are not subject to the 1st person hint so as not to disparage one's children or spouse. 2nd person hints are unsurprisingly used a bit more liberally because the pejorative sound isn't present, so human relations indicated in this way cannot be seen as rude. In this way, we can see the allusion in the following examples:
    \par
    \example{\proximal\boundbad bawan}{PROX.PEJ.dog}{My dog}
    \example{\mediodistal\boundtruevowel aamu}{MED.TRUE.mother}{Your Mother}

  \subsection{Bound Adjectives}
  Most words that map to English adjectives in \langname are verbs. However, a small, calcified group of adjectivals may prefix to nouns. All of these have fully regular verbal equivalents, but it common to use these adjectival prefixes to indicate not only their base meaning, but also frequently extended meanings. The below table elaborates all of the bound adjectives:
  \vertspace
  \begin{wrapfigure}{l}{0.55\textwidth}
    \begin{tabular}{|l|l|l|}
      \hline
      Prefix                   & Base Meaning & Common Extenstions   \\ \hline \hline
      \prefixtext{\boundbig}   & big          & strong, scary, old   \\
      \prefixtext{\boundsmall} & small        & generic diminutive   \\
      \prefixtext{\boundbad}   & bad          & crappy, stupid,      \\
      \prefixtext{\boundneg}   & un-          & without, reversal    \\
      \prefixtext{\boundtrue}  & true         & honest, pure         \\
      \prefixtext{\boundripe}  & ripe         & ready, tasty         \\
      \prefixtext{\bounddead}  & dead         & rotten, lost         \\ \hline
    \end{tabular}
  \end{wrapfigure}
  Only one adjectival prefix may be applied to a given noun. Furthermore, some, such as \prefixtext{\bounddead}, are exceedingly uncommon on inanimate nouns. The prefixes \prefixtext{\boundbig}, \prefixtext{\boundsmall}, and \prefixtext{\boundtrue} are frequently used in forming epitaphs for influencial people and spirits or deities in religious contexts. The prefixes \prefixtext{\boundripe} and \prefixtext{\boundbig} also are often used for placenames too. These prefixes do not reduplicate, and when a noun reduplicates for the associative plural, the prefix that may be applied from this group will attach at the very front of the noun before the first reduplicated syllable. As mentioned before, the \prefixtext{\boundbad} prefix is often used to denote 1st person possession, and the \prefixtext{\boundtrue} prefix is common to hint at second person possession. Note that some of the prefixes drop their vowel if the modified word starts with a vowel as well. In the table, they are the ones with the parenthetic vowels.
  \vertspace

  \subsection{Pronouns}
    \subsubsection{Introduction to some Binding Morphemes}
    \begin{wrapfigure}{r}{0.35\textwidth}
      \begin{tabular}{|l|l|}
        \hline
        Suffix            & Usage                        \\ \hline \hline
        Agentive Noun     & \suffixtext{\agtsuffix}      \\
        Animate Noun      & \suffixtext{\animsuffix}     \\
        Inanimate Noun    & \suffixtext{\inansuffix}     \\
        Temporal Past     & \suffixtext{\pastsuffix}     \\
        Temporal Non-Past & \suffixtext{\nonpastsuffix}  \\
        Locations         & \suffixtext{\locationsuffix} \\ \hline
      \end{tabular}
    \end{wrapfigure}
    Many of the pronouns in \langname are formed my mixing various binding morphemes. The ones specific to their sections will be introduced there, but there is a category of quasi-suffixes that denote generic nouns of various animacies, times, locations, and so on. The following table presents these suffixes:
    \par\vertspace

    \subsubsection{Demonstratives and the \prefixtext{\proximal}, \prefixtext{\mediodistal}, and \prefixtext{\distal} Prefixes}
    \begin{wrapfigure}{l}{0.85\textwidth}
      \begin{tabular}{|l|l|l|l|}
        \hline
                  & Proximal             & Mediodistal             & Distal             \\ \hline \hline
        Agentive  & \proximal\agtsuffix  & \mediodistal\agtsuffix  & \distal\agtsuffix  \\
        Animate   & \proximal\animsuffix & \mediodistal\animsuffix & \distal\animsuffix \\
        Inanimate & \proximal\inansuffix & \mediodistal\inansuffix & \distal\inansuffix \\ \hline
      \end{tabular}
    \end{wrapfigure}
    There are three levels of proximity distinguished by demonstratives in \langname. The proximal indicates things close to the speaker. The mediodistal denotes things close to the listener. And the Distal denotes things close to neither of them. These are made by compounding the deictic prefixes \prefixtext{\proximal}, \prefixtext{\mediodistal}, and \prefixtext{\distal} to the nominal suffixes presented in the last section. This presents a three by three split in demonstrative pronouns based on animacy and distance, seen in the table.
    \par
    The Demonstrative pronouns are purely pronominal; they are not used to directly modify a noun. That instead is covered by directly appending the diectic prefixes to the noun, such as \textbf{\proximal wawan} for "this dog" (note the initial \phonemic{b} mutation in \textbf{bawan}). This said, as will be seen, there do exist verbal counterparts for demonstrative pronouns.
    \par
    Further words can be derived with the demonstrative prefixes as well with other suffixes shown in the last section. The temporal and locative suffixes can be used to create words like \textbf{\proximal\locationsuffix} for "here" and \textbf{\distal\nonpastsuffix} for "a while from now". Of the ones that can be clearly produced \textbf{\proximal\pastsuffix} and \textbf{\proximal\nonpastsuffix} may seem to be basically the same. Generally, the latter one would be used for ongoing events or events that just happened while the former would represent things that very recently completed. In looser usage, they do overlap in usage a little bit.
    \vertspace

    \subsubsection{Interrogative Words and the \prefixtext{\interrogative} Prefix}
    All of the interrogative words are formed with the \prefixtext{\interrogative} prefix in some way. The basic pronouns equivalent to "who" or "what" attach the suffixes resulting in \textbf{\interrogative\agtsuffix}, \textbf{\interrogative\animsuffix}, and \textbf{\interrogative\inansuffix}. Particular animacy interrogative pronouns are used when the speaker knows or has an idea of which class a certain referent may fall into; however, if there is no clue as to the animacy the default is \textbf{\interrogative\agtsuffix}.
    \par
    Similar to the demonstratives, the interrogative pronouns cannot directly modify nouns. Instead, if someone wishes to express something like "which child" or "which birds" the \prefixtext{\interrogative} prefix is attached to the noun directly, such as \textbf{\interrogative kuuthar} and \textbf{\interrogative waani\animvowelplural} for "which child" and "which birds". Again, notice the lenis stop weakening and the plural suffix on \texttt{baani}, bird.
    \par
    In addition to the interrogative pronouns the \suffixtext{\pastsuffix}, \suffixtext{\nonpastsuffix}, and \suffixtext{\locationsuffix} suffixes, among others, may append to the \prefixtext{\interrogative} prefix to form basic interrogative words. In this way, \textbf{\interrogative\locationsuffix} results in the word for "where", \textbf{\interrogative\pastsuffix} is "when" for past events, and \textbf{\interrogative\nonpastsuffix} is "when" for either future events, ongoing events, or very very recent events.
    \par
    Furthermore, the interrogative pronouns like the demonstrative ones will have verbal equivalents.
    \vertspace

    \subsubsection{The Reflexive Pronoun \textbf{\reflexive}}
    There exists a single reflexive pronoun, \textbf{\reflexive}, which covers the rough meaning of the English \textit{-self}. The reflexive can be used for both local referents, within the same clause it is used, or in more remote situations, where it refers to things outside of its own clause. It also works in concord with other pronouns forming phrases like \textbf{\firstpn\space\reflexive} or \textbf{\proximal\agtsuffix\space\reflexive} which are roughly equivalent to \textit{myself} and \textit{himself} respectively, though these compounded phrases are relatively rare in common usage. Often one will see the reflexive pronoun incorporated into the verbs with the \infixtext{\reflexinc} or \infixtext{\reflexincweak} affixes fitting in the slot for incorporated nouns. This incorporation style is especially common for activities like bathing or feeding which one frequently does for one's own benefit.
    \vertspace

    \subsubsection{Indefinite}
    \subsubsection{Personal Pronouns}
    Personal pronouns are rarely used since their function is a mandatory part of verbal inflection. When they are used, it is normally to bring that particular person into focus during the discourse. The first and second person pronouns are fully distinct words. The singular versions are \textbf{\firstpn} and \textbf{\secondpn} and the plural versions are \textbf{\firstpn\agtvowelplural} and \textbf{\secondpn\agtconsplural}. In contrast, the third person pronouns simply are the demonstrative pronouns, proper to the animacy of the referent. The obviation split remains present when demonstratives are used in this way by means of deixis. Using agentive versions as a reference, the proximal third person is \textbf{\proximal\agtsuffix}, same as the proximal demonstrative, and the obviative is \textbf{\distal\agtsuffix}. In rare cases, something like \textbf{\mediodistal\agtsuffix} may be seen for objects grammatically possessed by germane actors, especially the first and second persons, but it is much more frequent to simply restate the object in some way. If it is especially backgrounded, it will likely be incorporated into the verb in some way.
    \vertspace

    \subsubsection{Personal Possession}
    Personal possession, in the general case, split between third person and non-third person classes. The first and second persons, both singular and plural, have contracted forms based on the \infixtext{\possessive} infix. The third person forms plainly use that infix with the demonstratives as might be expected. The first and second person singular possessives have the fused forms of \textbf{\firstsgposs} and \textbf{\secondsgposs} respectively, and the plural form are \textbf{\firstplposs} and \textbf{\secondplposs}. These fused forms are simply placed in front of the noun or noun phrase to be possessed. For example, \textbf{\firstsgposs\space kalu\inanvowelplural} would be \textit{my hands} and \textbf{\secondplposs\space dhalhan} for, using a colloquialism, \textit{y'all's bunch of fruit}.
    
  \subsection{Multiple Affixes on Single Nouns}
  The prior sections enumerated the nominal affixes that can be attached to nouns, but did not go into the fairly common occurence of a given noun having several affixes. Of the strategies enumerated above only the True Plural and the Associative Plural tend to be mutually exclusive, but this depends on the lexical meaning of the word. For the suffixes, the True Plural markers always come before the Obviative markers. When a partially reduplicated noun is affixed with a Bound Adjective, the adjective binds at the front of the reduplicated noun with no further consonant mutations.
  \vertspace

  \subsection{Numbers}
  \langname features a base 10 number system with a subbase of 5. The number for 5 appears to derive be related to the word for hand, and the number for 10 is derived from the word for a "pair." The numbers between 5 and 10 are created using a verbal preposition and \textbf{kalu}, "five." Each of these roughly mean something like "one on five." This compounding mechanism produces a fair deal of phonetic unstability. The general patter beyond \textbf{dalanta} for ten is to simply place the lesser valued portion after it, with the exception of \textbf{aku\tenbase} and \textbf{ithanyutaa dalanta} for eleven and twelve. For the multiples of ten, they are simply formed by prefixing \textbf{dalanta} with the value less than ten, with some phonetic reductions happening in this pattern as well, mostly from the \phonemic{d} reducing under analysis of being fused to the number before it. This all leads to some numbers being quite long. Moving forward, the word for hundred, \textbf{nunuma} seems to be derived from the word \textbf{numar} which means a small army or armed group, seemingly for the numbers desired for such small scale campaigns or defenses that may have been part of the life of the speakers of this proto-language. The word for thousand is derived from seemingly the same word, but with the \prefixtext{\boundbig} prefix on a reduced form. Given the age of the language, there is no word for zero, terms like "nothing," "none," or "absent" being used instead. It is also difficult to reconstruct numbers that exceed the value of ten thousand; it may have existed and seen usage, but such numbers cannot be convincingly supported. The basic numbers are presented in the table.
  \par
  As one may be able to ascertain, the normal method for constructing larger numbers is to place the multiple before the digit in order of largest digit to smallest. There is of course the exception in smaller value numbers. In this case, a small base is modified with a phrase that roughly means \textbf{X on Y} such as with \textbf{ari\fivebase} for eight. The values less than ten continue to use this five based method for higher values as can be seen in the table for 28. As an example of a very large number that would be cumbersome to format for the table, 6942 comes out as \textbf{aku\fivebase\space\thousand\space tu\fivebase\space nunuma tuju dalanta ithan}.
  \par
  Finally, the ordinal numbers are almost always derived from the cardinal number by suffixing \suffixtext{\ordnum} at the very end of the number. Two exceptions exist with \textbf{kupun} and \textbf{thipi} for first and second respectively. As the table details, these exceptions spread to any number that have one or two as the least significant value, such as in twenty-one and twenty-two.
  \newline\vertspace
  \begin{table}\centering
    \begin{tabular}{|l|l|l|l|}
      \hline
      Number & Cardinal                      & Ordinal                              & Notes \\ \hline \hline
      1      & aku                           & kupun                                & Cardinal and Ordinal Disjoint                   \\
      2      & ithan                         & thipi                                & Cardinal and Ordinal Disjoint                   \\
      3      & ari                           & ari\ordnum                           &                                                 \\
      4      & tuju                          & tuju\ordnum                          &                                                 \\
      5      & kalu                          & kalu\ordnum                          &                                                 \\
      6      & aku\fivebase                  & aku\fivebase\ordnum                  &                                                 \\
      7      & ithanyutaa kalu               & ithanyutaa kalu\ordnum               & Coda N and palatal merge                        \\
      8      & ari\fivebase                  & ari\fivebase\ordnum                  &                                                 \\
      9      & tu\fivebase                   & tu\fivebase\ordnum                   & \phonemic{juju} reduced to \phonemic{ju}        \\
      10     & dalanta                       & dalanta\ordnum                       & From the collective of \textbf{dantar} - a pair \\
      11     & aku\tenbase                   & aku\tenbase\ordnum                   & Imitates the five base numerals                 \\
      12     & ithanyutaa dalanta            & ithanyutaa dalanta\ordnum            & Imitates the five base numerals                 \\
      13     & dalantari                     & dalantari\ordnum                     & Slightly contracted "ten - three"               \\
      14     & dalanta tuju                  & dalanta tuju\ordnum                  &                                                 \\
      15     & dalanta kalu                  & dalanta kalu\ordnum                  &                                                 \\
      16     & dalanta aku\fivebase          & dalanta aku\fivebase\ordnum          &                                                 \\
      17     & dalanta ithanyutaa kalu       & dalanta ithanyutaa kalu\ordnum       &                                                 \\
      18     & dalanta ari\fivebase          & dalanta ari\fivebase\ordnum          &                                                 \\
      19     & dalanta tu\fivebase           & dalanta tu\fivebase\ordnum           &                                                 \\
      20     & ithan dalanta                 & ithan dalanta\ordnum                 &                                                 \\
      21     & ithan dalanta aku             & ithan dalanta kupun                  & Return of disjoint Ordinal \textbf{kupun}       \\
      22     & ithan dalanta ithan           & ithan dalanta thipi                  & Return of disjoint Ordinal \textbf{thipi}       \\
      ...    &                               &                                      &                                                 \\
      25     & ithan dalanta kalu            & ithan dalanta kalu\ordnum            &                                                 \\
      ...    &                               &                                      &                                                 \\
      28     & ithan dalanta ari\fivebase    & ithan dalanta ari\fivebase\ordnum    &                                                 \\
      29     & ithan dalanta tu\fivebase     & ithan dalanta tu\fivebase\ordnum     &                                                 \\
      30     & ari dalanta                   & ari dalanta\ordnum                   &                                                 \\
      40     & tuju dalanta                  & tuju dalanta\ordnum                  &                                                 \\
      50     & kalulanta                     & kalulanta\ordnum                     & reduction of sequential \phonemic{l} syllables  \\
      60     & aku\fivebase lanta            & aku\fivebase lanta\ordnum            & \textbf{kalulanta} continues till hundred       \\
      70     & ithanyutaa kalulanta          & ithanyutaa kalulanta\ordnum          &                                                 \\
      80     & ari\fivebase lanta            & ari\fivebase lanta\ordnum            &                                                 \\
      90     & tu\fivebase lanta             & tu\fivebase lanta\ordnum             & \phonemic{juju} reduces a syllable again        \\
      99     & tu\fivebase lanta tu\fivebase & tu\fivebase lanta tu\fivebase\ordnum & \phonemic{juju} reduces a syllable again        \\
      100    & nunuma                        & nunuma\ordnum                        & Related to "numar" -- "small army"              \\ 
      200    & ithan nunuma                  & ithan nunuma\ordnum                  &                                                 \\ 
      300    & ari nunuma                    & ari nunuma\ordnum                    &                                                 \\ 
      400    & tuju nunuma                   & tuju nunuma\ordnum                   &                                                 \\ 
      500    & kalu nunuma                   & kalu nunuma\ordnum                   &                                                 \\ 
      1000   & \thousand                     & \thousand\ordnum                     & "big" prefix on reduced hundred                 \\ 
      8000   & ari\fivebase\space\thousand   & ari\fivebase\space\thousand\ordnum   &                                                 \\ \hline
    \end{tabular}
  \end{table}
