\section{Nouns}
Nouns in \langname are clearly an inflectional category of words, however the degree of morphology attached to nouns is fairly light. All nouns have an implicit animacy that fits into a loose hierarchy which has some minor effects on how nouns may be inflected. As will be seen, this animacy hierarchy has some importance in the verbal morphology as well. Furthermore, third person nouns are marked for obviation for various reasons ranging from sentence level distinctions, to thematic roles, to the general flow of discourse.

\subsection{The Animacy Hierarchy}
\langname displays a three level animacy hierarchy. Though mostly the same throughout, the different levels are treated differently in minor ways through the language, and the animacy system plays a significant role in verbal morphology. The 3 classes distinguish percieved inanimacy, animacy, and sentience, they are listed here with some examples.
\vertspace
  \begin{wrapfigure}{l}{0.65\textwidth}
    \begin{tabular}{|l|l|}
      \hline
      Class           & Examples                               \\ \hline \hline
      Agentive        & Humans, Spirits, dogs, horses, hawks   \\
      Animate         & Small birds, reptiles, insects, plants \\  % predates domesticated livestock
      Inanimate       & Rocks, water, tools, housing, ideas    \\ \hline
    \end{tabular}
  \end{wrapfigure}
  This table gives a rough illustration of how agency typically functions in \langname. That said, certain things may end up higher or lower depending on general or particular perceptions of agency. For example, if a reptile of some sort is seen as threatening or scheming it may be sporadically elevated to the agentive class. Lowering animacy can happen, but is exceedingly rare. Though one might think it would be useful for a rude insult, that too is very very rare. The culture of insults works rather differently.

\subsection{Common Nominal Particles and Patterns}
There a handful of general use particles and patterns in \langname. With the exception of a couple, they always join at the end on the noun stem, and several may stack on top of each other.

  \subsection{True Plurals with \suffixtext{\agtconsplural}, \suffixtext{\animplural}, and \suffixtext{\inanplural}}
  Plurals can be denoted on all nouns, but are only mandatory on agentive nouns and personal pronouns. The plural suffixes are \suffixtext{\agtplural}, \suffixtext{\animplural}, and \suffixtext{\inanplural} for the agentive, animate, and inanimate noun classes respectively. For agentive nouns, this surfaces as \suffixtext{\agtvowelplural} after vowels and as \suffixtext{\agtconsplural} after consonants. Some simple examples follow:

  \example{ramaa\agtvowelplural}{spirit.PL}{spirits}

  \subsection{Associative Plurals via Partial Reduplication}
  Associative Plurals are common in \langname, and are used to indicate groups or collections of nouns that are the same or similar. For all noun classes, the associative plural is denoted by reduplicating the first syllable, with some rules based on the structure of the initial syllable. For syllables with an initial consonant, only the initial consonant-vowel sequence is reduplicated with coda consonants ignored. For words with initial vowels, if the vowel is short it will become long. However, if a word has an initial long vowel, reduplication is not possible. This process is never applied to given names, and there is an aversion towards its use for very specific individuals.
  Certain consonants also mutate in the reduplication process. In particular, all the lenis stops lenite to approximants or laterals. This is enumerated in the table below:

  \vertspace
  \begin{tabular}{|l|l|l|l|}
    \hline
    Consonant   & Mutated    & Base Word        & Reduplicated Example         \\ \hline \hline
    \textbf{b}  & \textbf{w} & Baani - Bird     & Baawaani - A flock           \\
    \textbf{d}  & \textbf{l} & Dugya - Mountain & Dulugya  - Mountain Range    \\
    \textbf{dh} & \textbf{l} & Dhan  - A Fruit  & Dhalan   - A Bunch of Fruits \\
    \textbf{gy} & \textbf{y} & Gyuul - A Bee    & Gyuuyuul - A Swarm of Bees   \\
    \textbf{g}  & \textbf{w} & Gara  - Sand     & Gawara   - Desert            \\ \hline
  \end{tabular}
  
  \vertspace
  \example{Yama -- Yayama}{tuber -- RED.tuber}{tuber -- a bunch of tubers}

  \subsection{Obviation with \suffixtext{\agtconsobv} and \suffixtext{\inanobv}}
  Obviation is a marking system that is used to distinguish thrid person nouns that might otherwise be confused.  For simple sentences, this allows for clarity in who effects the verb on the other and the degree of volition for lexcially reciprocal verbs. At a discourse level, it keeps competing topics clear. Both agentive and animate verbs share a suffix for indicating the obviative, \suffixtext{\agtobv}, which is \suffixtext{\agtvowelobv} after vowels and \suffixtext{\agtconsobv} after consonants. Inanimate nouns are always marked with \suffixtext{\inanobv}.
  It should be noted, obviation is only marked when the two nouns are of the same animacy or in certain cases where a lower animacy noun exerts on a higher one. In the general case, if two third person nouns of different animacy are present, the higher animacy is assumed to be proximate and the lower obviative.

  \subsection{Possession with the \infixtext{\possessive}}
  Possession in \langname is marked via the infix \infixtext{\possessive}. This does not vary based on the animacy of the constituent nouns. This infix is however not used when possessive prouns may fill the role. Furthermore, when two third person nouns, including pronouns, are the constituents of a possessive noun phrase, the possessed noun will always be marked as obviative

  \example{Aamu ini bawa\agtvowelobv}{Mom POSS dog.OBV}{Mom's dog}

  \subsection{Bound Adjectives}
  Most words that map to English adjectives in \langname are verbs. However, a small, calcified group of adjectivals may prefix to nouns. All of these have fully regular verbal equivalents, but it common to use these adjectival prefixes to indicate not only their base meaning, but also frequently extended meanings. The below table elaborates all of the bound adjectives:
  \vertspace
  \begin{wrapfigure}{l}{0.55\textwidth}
    \begin{tabular}{|l|l|l|}
      \hline
      Prefix             & Base Meaning & Common Extenstions   \\ \hline \hline
      \prefixtext{ma}    & big, large   & strong, scary, old   \\
      \prefixtext{kya}   & bad          & pejorative, false    \\
      \prefixtext{iwi}   & small        & cute, beloved, young \\
      \prefixtext{iba}   & good         & functional, favored  \\
      \prefixtext{yuu}   & true         & honest, pure         \\
      \prefixtext{ggu}   & ripe         & ready, balanced      \\
      \prefixtext{nhai}  & dead         & rotten, lost         \\ \hline
    \end{tabular}
  \end{wrapfigure}
  Only one adjectival prefix may be applied to a given noun. Furthermore, some, such as \prefixtext{nhai} and \prefixtext{yuu}, are exceedingly uncommon on inanimate nouns. The prefixes \prefixtext{ma}, \prefixtext{iwi}, and \prefixtext{yuu} are frequently used in forming epitaphs for influencial people and spirits or deities in religious contexts. The prefixes \prefixtext{iba} and \prefixtext{ggu} also is often used for placenames too.

  \subsection{Personal Pronouns}
  Personal Pronouns differentiate on the basis of person and number. In the case of third person referents, there is also a distinction on the basis of animacy and obviation. This said, the usage of personal pronouns is very infrequent because the verbal morphology normally provides that information in a sentence. For full clarity here, first and second person referents do not interact with obviation. When the topicality of one of them needs to be lowered, it is almost always done by marking the verb with the inverse which will be covered in later sections.
  \vertspace
  \begin{wrapfigure}{l}{0.55\textwidth}
    \begin{tabular}{|l|l|l|}
      \hline
      Person                  & Singular & Plural              \\ \hline \hline
      1st                     & Gyaa     & Gyaa\agtvowelplural \\
      2nd                     & Uuru     & Uuru\agtvowelplural \\
      3rd Agent               &          &        \\
      3rd Agent Obviative     &          &        \\
      3rd Animate             &          &        \\
      3rd Animate Obviative   &          &        \\
      3rd Inanimate           &          &        \\
      3rd Inanimate Obviative &          &        \\ \hline
    \end{tabular}
  \end{wrapfigure}

  \subsection{Multiple Affixes on Single Nouns}
  The prior sections enumerated the nominal affixes that can be attached to nouns, but did not go into the fairly common occurence of a given noun having several affixes. Of the strategies enumerated above only the True Plural and the Associative Plural tend to be mutually exclusive, but this depends on the lexical meaning of the word. For the suffixes, the True Plural markers always come before the Obviative markers. When a partially reduplicated noun is affixed with a Bound Adjective, the adjective binds at the front of the reduplicated noun with no further consonant mutations.
