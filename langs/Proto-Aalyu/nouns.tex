\section{Nouns}
Nouns in \langname are clearly an inflectional category of words, however the degree of morphology attached to nouns is fairly light. All nouns have an implicit animacy that fits into a loose hierarchy which has some minor effects on how nouns may be inflected. As will be seen, this animacy hierarchy has some importance in the verbal morphology as well. Furthermore, third person nouns are marked for obviation for various reasons ranging from sentence level distinctions, to thematic roles, to the general flow of discourse.

\subsection{The Animacy Hierarchy}
\langname displays a three level animacy hierarchy. Though mostly the same throughout, the different levels are treated differently in minor ways through the language, and the animacy system plays a significant role in verbal morphology. The 3 classes distinguish percieved inanimacy, animacy, and sentience, they are listed here with some examples.
\vspace{3mm}
  \begin{wrapfigure}{l}{0.65\textwidth}
    \begin{tabular}{|l|l|}
      \hline
      Class           & Examples                               \\ \hline \hline
      Agentive        & Humans, Spirits, dogs, horses, hawks   \\
      Animate         & Small birds, reptiles, insects, plants \\  % predates domesticated livestock
      Inanimate       & Rocks, water, tools, housing, ideas    \\ \hline
    \end{tabular}
  \end{wrapfigure}
  This table gives a rough illustration of how agency typically functions in \langname. That said, certain things may end up higher or lower depending on general or particular perceptions of agency. For example, if a reptile of some sort is seen as threatening or scheming it may be sporadically elevated to the agentive class. Lowering animacy can happen, but is exceedingly rare. Though one might think it would be useful for a rude insult, that too is very very rare. The culture of insults works rather differently.

\subsection{Common Nominal Particles and Patterns}
There a handful of general use particles and patterns in \langname. With the exception of a couple, they always join at the end on the noun stem, and several may stack on top of each other.
  \subsubsection{True Plurals}
  Plurals can be denoted on all nouns, but are only mandatory on agentive nouns and personal pronouns. The plural suffixes are \particletext{\agtplural}, \particletext{\animplural}, and \particletext{\inanplural} for the agentive, animate, and inanimate noun classes respectively. For agentive nouns, this surfaces as \particletext{\agtvowelplural} after vowels and as \particletext{\agtconsplural} after consonants. Some simple examples follow:

\example{ramaa\agtvowelplural}{spirit.PL}{spirits}

