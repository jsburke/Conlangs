\section{Phonology}
\langname has an abundance of consonants that fills their space rather thoroughly. The vowels, conversely, are less than a typical language, and contrast only in length. The syllable structure is fairly simple, with open syllables being preferred overall. There are no tonal distinctions, and the words are mora timed. For the charts in the vowel and consonant sections, the romanization is provided in angle brackets to the right of each phoneme. The methodology for deriving this scheme will be discussed in a later section within the phonology with the main justification being that internal consistency and ease of use dominated over matching with the IPA or favoring an Anglocentric use of the alphabet.

  \subsection{Vowels}
     \begin{wrapfigure}{l}{0.55\textwidth}
       \begin{tabular}{|l|l|l|l|}
         \hline
                 & Front              & Central            & Back               \\ \hline \hline
         High    & i, \tlongi <i, ii> &                    & u, \tlongu <u, uu> \\
         Low     &                    & a, \tlonga <a, aa> &                    \\ \hline
       \end{tabular}
     \end{wrapfigure}
     \langname features three vowel qualities at the extremes of the vowel space. Vowels arise both as long and short variants, but long vowels are far fewer since they rarely surface outside of the first syllable of stems, especially in nouns. The most significant departure from this is that all verbal stems in the plain form have a \phonemic{\tlonga} of some kind at their end. The vowels are synchronically very stable with no notable changes in quality regardless of length or environment. The long vowels are not distinguishable from a series of two short vowels. Also, diphthong like vowel-vowel sequences like \phonemic{ai} and \phonemic{iu} are extremely rare.

  \subsection{Consonants}
  \begin{wrapfigure}{r}{0.7\textwidth}
    \begin{tabular}{|l|l|l|l|l|l|}
      \hline
                    & Labial & Dental       & Retroflex    & Palatal        & Velar       \\ \hline \hline
        Fortis      & p <p>  & \dentalt <t> & \retrot <th> & c <ky>         & k <k>       \\ 
        Lenis       & b <b>  & \dentald <d> & \retrod <dh> & \lenisc <gy>   & g <g>       \\ 
        Nasal       & m <m>  & \dentaln <n> & \retron <nh> & \palataln <ny> & \engma <gg> \\ 
        Rhotic      &        & \dentalr <r> & \retror <rh> &                &             \\ 
        Lateral     &        & \dentall <l> & \retrol <lh> & \palatall <ly> &             \\ 
        Approximant &        &              &              & j <j>          & w <w>       \\ \hline
    \end{tabular}
  \end{wrapfigure}
  \langname makes use of five places of articulation, in which three manners are fully present. Being a proto-language, the difference between fortis and lenis stops is not necessarily one of voicing, but could have also been aspiration or another mechanism which caused consistent differences to be related in daughter languages; the choice of using tenuis and voiced graphemes here reflects only their ease of use. In daughter languages, regardless of how they surface, the lenis consonants are much more prone to diachronic changes than either the fortis or nasal consonants.
  \par
  The lenis consonants are further restricted to only word initial positions. Even within the protolanguage, it seems amply evident that when non-nasal sounds precede them in a word, they lenite to other sounds, which will be detailed later. However, when nasal sound occurs before a lenis consonant, such as with a prefix that has a coda nasal, the lenis consonant remains.
  \par
  A further note is that the dental rhotic is expected to have been rendered as a tap consonant rather than a trill, but the two could have been interchangeable.
  \par
  Furthermore, as presented in the following sections, there exist 3 underspecified syllable coda consonants which shall be rendered as \textbf{N}, \textbf{L}, and \textbf{R}. These respectively are homorganic coda nasals, laterals, and rhotics. 

  \subsection{Coda Consonant Realizations}
  The coda consonants are very underspecified and in general homorganic to the place of articulation of the following consonant, if there is one. The nasal shows the most mutability while the rhotic shows the least which is reflective of the number of points of articulation in the language overall.
  When these occur word medially, they take on the place of articulation of the following consonant. For example, if \textbf{L} is before <ny> then it will surface as \phonemic{\palatall}. Naturally, this means that \textbf{N} always mutates. When the rhotic and lateral can match place of articulation they do, but when they cannot, the dental version surfaces.
  When a coda consonant is present at the end of a word, the lateral and rhotic take of their dental forms, but the nasal takes on its velar form <gg>. If suffixes attach after it, they will once again become homorganic as described. If the suffix presents an initial vowel, then each of the coda consonants take on their dental form and become phonetically an initial consonant for the suffix.

  \subsection{Syllable Structure and Timing}
  The syllable structure is simply \texttt{(C)V(T)} where \texttt{T} is one of the homorganic coda consonants. Of these, the nasal \textbf{N} is more common than the other two. This means that there are four distinct syllable types, where the content affects how long it is. Sequences of \texttt{VV}, two sequential short vowels, has the same timing as a long vowel. However, short vowel clusters like that where the two differ in quality are extremely rare. For that reason, they're mentioned here but not explicit in the table.
  \vertspace
  \begin{wrapfigure}{l}{0.35\textwidth}
    \begin{tabular}{|c|c|}
      \hline
      Syllable        & No. Morae \\ \hline \hline
      \texttt{(C)V}   & 1         \\
      \texttt{(C)V:}  & 2         \\
      \texttt{(C)VT}  & 2         \\ 
      \texttt{(C)V:T} & 3         \\ \hline
    \end{tabular}
  \end{wrapfigure} 
  This timing system is fairly direct. A short vowel presents a single mora, a long vowel two, and coda consonants add an extra onto the syllable. Though the coda consonants take a full mora, they never form the nucleus of any phonemic syllable; however, the use of such sounds when thinking of what to say is not uncommon. All that said, some speakers and regions seem to prefer making some word initial patterns like \phonemic{um} a syllablic nasal or \phonemic{ar} a syllabic rhotic, especially at the beginning of words. This is not universal, however, but the pattern is definitely evident in daughter languages.\par

  \subsection{Romanization Scheme}
  The Romanization scheme is meant to be as internally consistent as possible and as quick to learn as possible as well, so this section will primarily present this reasoning with justifications for the inconsistencies present. In the interest of ease of use, this scheme was chosen to prefer digraphs over diacritics. It is also largely a phonemic mapping, but there are exceptions allowed for ease.\par
  The vowels are simply echoes of the IPA in quality, and length is noted by doubling the vowel. This doubling provides a dual convenience. It clearly indicates the underlying moraic system that would be less transparent with something like macrons. It also is fairly intuitive since many languages, both natural and constructed, already use such a scheme.\par
  The consonants probably require a little more effort to internalize. First, as the consonant section notes, the distinction between the fortis and lenis stops is not necessarily one of voicing, but the dividing line between voiced and unvoiced consonants in the Latin alphabet has been leveraged to this effect, which is not to far a cry from Hanyu Pinyin which uses it for aspiration. The retroflex series of consonants follows a simple pattern of appending a <h> to the associated dental sound. This was primarily done because <h> is not used elsewhere in \langname because of the lack of fricatives, so it is fairly clear that a digraph is indicated and syllable boundaries are not ambiguous. The palatal series follows a similar pattern of appending an otherwise unused graph, but is a bit less organized given the nature of the Latin alphabet and how palatal consonants are seemingly rendered in ad hoc manners in it. The graph <y> is appended to either the associated velar or dental consonant, with the exception of \phonemic{j} which is simply rendered as <j>. This latter choice was made so that coda nasals and laterals followed by \phonemic{j} could be easily distinguished from the palatal nasal and lateral without needing to resort to options like apostrophes to indicate syllable boundaries. The <ly> and <ny> digraphs were chosen both for familiarity and because appending to a velar series graph would either be impossible or cumbersome. The fortis and lenis palatal stops, however, append to the velar series. This is mainly done as an aesthetic measure so that <t> and <d> won't appear to be in abundance and in hopes it may be more intuitive than <ty> or <dy> digraphs may be.\par
  The velar nasal, \phonemic{\engma}, is represented as <gg>. This is admittedly quite inelegant and rather non-intuitive. Had <g> not been taken by the lenis velar stop and <ng> not possible at a syllable boundary, both would have been preferred. However, since neither of those two were easy to use and other options like <q> and <\textipa{\~g}> were either less intuitive or more cumbersome to input, <gg> has been selected for clarity and some degree of internal consistency.\par
  Finally, the coda consonants are almost always represented by the dental graph of their associated type, thus <n, l, r> for the nasal, lateral, and rhotic, respectively. An inconsistency with the nasal is tolerated. When it occurs before the labial consonants \phonemic{p, b, m}, it may be written as <m>; however this is not mandatory. This document will strive to use <n> in all these cases.\par

  \subsection{Common Sound Mutations}
  Very many processes in \langname trigger phonetic changes. Rather than leave those scattered through this document, they will be summarized here. The grammatical rules that cause these patterns will not be explained here, but they should be easily searchable in this document.

  \subsubsection{Lenis Consonant Mutations}
  The lenis series of stops is notable for only being nativley attested word initially, and they often reduce when other sounds are affixed to the front of the word. If a vowel, coda \textbf{R}, or coda \textbf{L}, but not \textbf{N}, ends up in front of a lenis consonant, it will reduce as the table below indicates, which shows examples from partial reduplication of nouns:

  \vertspace
  \begin{tabular}{|l|l|l|l|}
    \hline
    Consonant   & Mutated     & Base Word        & Reduplicated Example         \\ \hline \hline
    \textbf{b}  & \textbf{w}  & Baani - Bird     & Baawaani - A flock           \\
    \textbf{d}  & \textbf{l}  & Dugya - Mountain & Dulugya  - Mountain Range    \\
    \textbf{dh} & \textbf{lh} & Dhan  - A Fruit  & Dhalhan   - A Bunch of Fruits \\
    \textbf{gy} & \textbf{j}  & Gyuul - A Bee    & Gyuujuul - A Swarm of Bees   \\
    \textbf{g}  & \textbf{w}  & Gara  - Sand     & Gawara   - Desert            \\ \hline
  \end{tabular}

