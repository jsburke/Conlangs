\section{Verbs}
Verbs in \langname are a highly complex and highly inflecting category. Verbs inflect for several important topics like both subject and object, past versus non-past tenses, and various aspects and moods. It also features wide usage of noun incorporation and many adverbial like constructions formed from non-finite verb forms. The verbs depend heavily on the inherent animacy hierarchy discussed in the nouns section.
\par
Verbs also cover what in English and many other languages are considered adjectives. In this way, the verb meaning \textit{red} should be translated as a predicate like "is red." Similarly, what would be considered more classical verbs to many have inflections that make them more explicitly adjectival so that the word for \textit{rotting} in "the rotting tree" fits properly into that noun phrase.
\par
On the topic of adverbial inflected verbs, \langname provides inflections such that \textit{after going} in a sentence like "he slept for the night after going to the far camp" are inflections of the verb itself rather than phrasal constructions. In this same way, the \textit{and} in something simple like "He ate yams and drank mead" is shown by inflecting the verb for "to eat".
\par
This hopefully gives the true impression that verbs are a very complex category. In hopes of providing as much clarity as possible, the following sections will cover a given topic in verbs one at a time and try to build upon previously read material.

\subsection{Basic Intro and Verbal Implicits}
Verbs in the attached glossary and the larger lexicon will be provided in the most uninflected form possible. Some verbs, such as "to give" will be underinflected in these entries, rendering them ungrammatical; this is because of the verb expecting some kind of object to be contextualized, most often through noun incorporation of some kind.
\par
The basic dictionary style entry of a given verb will always end with the vowel sequence \textbf{aa}. This implies at least a 3rd person subject and possibly a 3rd person object in the present tense with a plain informative mood for something that is most likely ongoing. Thus the dictionary entry for "to see" would be \textbf{laniyaa}, which strictly translated would be something like "he sees it" or something similar with other 3rd person actors.
\par
Implicit in this is an animacy hierarchy that governs expected subjects and objects. This ordering is 2nd > 1st > 3rd > 3rd obviative. Within the 3rd person categories in the implict animacies, where agentive is more thematic than animate which is more thematic than inanimate. This means for two thrid person nouns of different animacies, the higher animacy class noun is the default subject. 1st and 2nd person actors are marked on the verb explicitly with affixes, and an inversion affix is used when the expected subject hierarchy is reversed, such as when a 1st person actor exerts a verb on a 2nd person actor.

\subsection{Inflecting for 1st and 2nd Persons with the \suffixtext{\verbfirst}, \suffixtext{\verbsecond}, and \suffixtext{\verbonetwo} Affixes}
Verbs take small suffixes that indicate persons beyond the default 3rd person actors. If the verb has a first person subject, it will have the \suffixtext{\verbfirst} affix right after the verb stem. Similarly, for a second person subject, the \suffixtext{\verbsecond} will attach at the same spot as the first person. Finally, if there is a second person subject and first person object, the \suffixtext{\verbonetwo} affixes in the same spot again. For the \suffixtext{\verbfirst} and \suffixtext{\verbsecond} affixes that only indicate a single actor on the verb, if the verb is transitive, it is assumed that the object is some 3rd person actor that may be indicated plainly in a separate word or via noun incorporation on the verb itself. One their own, these affixes can only reflect the default thematic hierarchy for subjects.

\subsection{Changing Verb Subjects by Inversion with \infixtext{\verbinv}}
In the previous section, the affixes to change who the subject of a verb is was explained; however those affixes never allow for a third person subject and a first or second person object or a first person subject and a second person object because they concur with the verb's implicit actor ordering scheme, which was covered in the Intro and Implicits section. In order to allow for these actor arrangements that oppose the default ordering system, the inversion affix, \suffixtext{\verbinv}, is placed right after the verb stem right before any explicit person markers. Using the earlier verb example, \textbf{laniyaa} means 'he sees it.' Placing a person affix on that same verb we arrive at \textbf{laniyaa\verbfirst} results in 'I see him'. Adding the inversion marker to the mix results in \textbf{laniyaa\verbinv\verbsecond} means 'he sees you.' Covering the final corner in this conflux uses the double person marker and inversion resulting in \textbf{laniyaa\verbinv\verbonetwo} means 'I see you.'
