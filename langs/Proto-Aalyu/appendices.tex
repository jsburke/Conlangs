\addappendix{Appendix Foreword}{The following are supplementary sections to the \langname grammatical description for the convenience of the reader. There are several topics and listings that the reader might want to use while reading without needed another documents open at the same time. It includes an abbreviated lexicon, limited to words used in the grammatical description, a listing of grammatical affixes and particles out of context, some cultural notes and tidbits, and extended passages in \langname with translation into English. Each of the appendices to follow will be given a brief description here in the foreword.
\vertspace

\noindent\begin{tabular}{ll}
  Appendix                              & Description \\ 
  \textbf{Reference Lexicon}            & A listing of all the \langname words used in this document \\
  \textbf{Particle Listing}             & A listing and description of particles and affixes         \\   
  \textbf{\langname Creation Story}     & The creation story in \langname and English                \\
  \textbf{\langname The Taming of Dogs} & The story of man and dog becoming friends                  \\
\end{tabular}}

% quick and dirty command for lexicon
\newcommand{\lexentry}[2]{\textbf{#1} -- #2\newline}

\addappendix{Reference Lexicon}{
\begin{multicols}{3}
\noindent\lexentry{aamu}{Mother}
\lexentry{aku}{one}
\lexentry{ari}{three}
\lexentry{baani}{bird}
\lexentry{bawa}{dog}
\lexentry{\reflexive}{"self"} % binya
\lexentry{bunpa}{Head}
\lexentry{dhan}{a fruit}
\lexentry{dukya}{mountain}
\lexentry{ithan}{two}
\lexentry{gara}{sand}
\lexentry{gyaal}{bee}
\lexentry{\firstpn}{I, me}    % gyuu
\lexentry{\firstpn\agtvowelplural}{I, me}    % gyuu
\lexentry{\secondpn}{I, me}    % gyuu
\lexentry{\secondpn\agtvowelplural}{I, me}    % gyuu
\lexentry{raama}{spirit}
\lexentry{yama}{tuber, potato, yam}
\end{multicols}
}
